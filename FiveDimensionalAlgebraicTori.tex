\documentclass[12pt]{article}

\usepackage{appendix}
\usepackage{graphicx}
\usepackage{amsmath}
\usepackage{amsthm}
\usepackage[byname]{smartref}
\usepackage{hyperref} 
\usepackage{txfonts}
\usepackage{tocloft}
\usepackage{tikz}
\usepackage{longtable}
\usepackage{xcolor}
\usepackage{listings}
\usepackage{calligra}
%\usepackage{mathtools}
\usetikzlibrary{positioning}
\usepackage{tikz-qtree,tikz-qtree-compat}
\usetikzlibrary{trees}
\usetikzlibrary{arrows,positioning,automata,shadows,fit,shapes}
\usepackage{float}
\usepackage{amssymb}
\usepackage{tikz-cd}
\usepackage[english]{babel}
%\usepackage[table]{xcolor}
\usepackage{algorithmic}
\usepackage{algorithm}
\usepackage{authblk}
\usepackage{tikz-cd}
\usepackage{chngcntr}
\renewcommand{\algorithmicrequire}{\textbf{Input:}}
\renewcommand{\algorithmicensure}{\textbf{Output:}}
%%%%%%%%%%%%%%%%%%%%%%%%%%%%
\usepackage{graphicx}
\usepackage{amsmath}
\usepackage{amsfonts}
\usepackage{mathrsfs}
%\usepackage[a4paper, total={8in, 10in}]{geometry}


\hypersetup{
%	unicode = true,
	colorlinks = true,
	citecolor = blue,
%	filecolor = black,
	linkcolor = blue,
	urlcolor = blue,
%	pdfstartview = {FitH},
}



\bibliographystyle{amsplain}

\theoremstyle{plain}
\newtheorem{theorem}{Theorem}
\newtheorem{lemma}[theorem]{Lemma}
\newtheorem{corollary}[theorem]{Corollary}
\newtheorem{proposition}[theorem]{Proposition}
\theoremstyle{definition}
\newtheorem{definition}[theorem]{Definition}
\newtheorem{conjecture}[theorem]{Conjecture}
\newtheorem{example}[theorem]{Example}
\newtheorem*{remark}{Remark}
\newtheorem{note}[theorem]{Note}


\newcommand{\N}{\ensuremath{\mathbb{N}}}
\newcommand{\Z}{\ensuremath{\mathbb{Z}}}
\newcommand{\Q}{\ensuremath{\mathbb{Q}}}
\newcommand{\C}{\ensuremath{\mathcal{C}}}
\newcommand{\Po}{\ensuremath{\mathcal{P}}}
\newcommand{\G}{G}
\newcommand{\glat}{$G$-lattice}
\newcommand{\tand}{\ensuremath{\,\,\, \text{and} \,\,\,}}
\newcommand{\exactseq}[1]{\ensuremath{0 \longrightarrow M_{#1} \longrightarrow L_{#1} \longrightarrow \Z \longrightarrow 0}}
\newcommand{\exactseqs}[1]{\ensuremath{0 \longrightarrow M_{#1} \longrightarrow L_{#1} \longrightarrow \Z^{-} \longrightarrow 0}}
\newcommand*{\QEDA}{\hfill\ensuremath{\blacksquare}}

\newcommand{\todo}[1]{{\bf {\color{red}To do: #1}} }

\title{Five Dimensional Algebraic Tori}



\begin{document}
\maketitle
\abstract{ 
  The rationality problem for algebraic tori of dimentions 2, 3 and 4 (with a few unknown cases in dimension 4) is solved. 
  The purpose of this work is to solve rationality problem for 5 dimensional algebraic tori in an experimental way. In order to 
  do so, we have studied the associated character lattices of the mentioned 
  algebraic tori. For each indecomposable character lattice $L$, either we identify the lattice as 
  an associated one to a root system (of which the rationality of its corresponding 
  algebraic torus is known) or we find a reduced component of $L$ so that we can 
  relate rationality of the associated algebraic torus to lower dimensions. Using 
  these two main methods from \cite{Nicole1}, we solve rationality problem in some cases. 
  For those tori with a decomposable character lattice, we use the known results on the rationality 
  of their decomposition components to determine their rationality.
  }
  
  
\section{Introduction}
An interesting problem in algebraic geometry is the rationality problem, i.e. 
for a given algebraic variety, determine if it is rational or not. Naturally 
this is a difficult problem to solve. Hence it makes sense to consider some 
relaxed notions of rationality. 

An algebraic $F$-variety $X$ (for a field $F$) is called rational if there exists a birational map 
form $X$ to $\mathbb{A}^n$ for some $n$. Algebraically, $X$ is rational if 
its rational function field is $F$-rational, i.e. it is a purely 
transcendental extension of $F$. $X$ is called stably rational if $X \times 
\mathbb{A}^n$ is rational for a non-negative integer $n$. Similarly one can say $X$ is stably rational if 
a purely transcendental extension of the rational function field of $X$ is 
rational. Finally $X$ is called unirational if the rational function field of 
$X$ is a subfield of a purely transcendental extension of $F$. Geometrically 
this means there exists a dominant rational map from $\mathbb{A}^n$ to $X$ for
some $n$. It is worth mentioning that there is a family of varieties which is
called retract rational which contains the stably rational ones and is contained in 
unirational family. However since we have nothing to do with retract rational tori we 
do not present their definition (see \cite{Hoshi} for more detail).

It can be seen that rationality implies stable rationality and the later implies 
unirationality. It is well-known that none of the converses of these implications
hold in general for varieties. 

Assume $X$ is a quasi projective $F$-variety. An $F$-variety $Y$ is called an 
$F$-form of $X$ if $X \otimes_F \bar{F} \cong Y \otimes_F \bar{F}$ where $\bar{F}$ is the algebraic closure of $F$. A $d$-dimensional algebraic 
$F$-torus is a group $F$-scheme such that it is an $F$-form of $\mathbb{G}_m^d$ (see \cite[Section 1.3]{Voskresenskii} for more detail).

There is a one-to-one correspondence between the isomorphism classes of 
$n$ dimensional algebraic $F$-tori and representations of $G_F = \mathrm{Gal}(F_s/F)$ 
into $\mathrm{GL}(n,\Z)$, where $F_s$ is the separable closure of $F$. We know that
 there exists a finite Galois extension of $K/F$, such that 
$T \otimes_F K \cong \mathbb{G}_{m,K}^n$. The smallest such Galois extension 
(resp. Galois group) is called the splitting field (resp. splitting group)
 of $T$.
 
The isomorphism classes of $n$ dimensional algebraic $F$-tori are in bijection 
with conjugacy classes of finite subgroups of $\mathrm{GL}(n,\Z)$. By a theorem of Jordan
 we know that the number 
of these finite subgroups up to conjugacy is finite. If $G \leq \mathrm{GL}(n,\Z)$ then 
the standard lattice, $L = \langle e_i: 1 \leq i \leq n \rangle_\Z$ where $e_i = [\delta_{1i}]$,
with the action (right multiplication) of $G$ defines a $G$-lattice $L_G$ ($\delta_{ij}$ is the Kronecker delta). 
Assume $K/F$ is a finite Galois extension with $G = \mathrm{Gal}(K/F)$. Now $T_G = \mathrm{Spec}(K[L_G]^G)$
is an algebraic torus. Note that conjugate groups correspond to isomorphic 
lattices and hence are associated to isomorphic algebraic tori. Conversely for 
a given $n$-dimensional algebraic torus $T$, which is defined over $F$ and split 
by $K$, its character group is a $G$-lattice of rank $n$ and hence associates 
to a conjugacy class of subgroups of $\mathrm{GL}(n,\Z)$.  

It is well-known that all algebraic tori are unirational. There are unirational algebraic tori which 
are not stably rational. Interestingly there exists a famous conjecture by Voskeresenskii,
\begin{center}
 ``any stably rational algebraic torus is rational''.
 \end{center} To the 
best of the authors knowledge the conjecture is still open. A substantional amount of work in rationality problem for algebraic tori, is done by Voskeresenskii, Endo, Miyata, Saltman, 
Sansuc and colliet-thelene. Their approach to address the problem was the
correspondence between algebraic tori and $G$-lattices. Their efforts led to 
characterization of stably rational and retract rational algebraic tori based 
on their character lattices.

Voskeresenskii proved that any two dimensional algebraic torus is rational. 
In early 90's Kunyavski classified three dimensional algebraic tori up to 
rationality. He showed that except for 15 three dimensional algebraic tori 
which are not retract rational, the rest of them are rational.

In 2012 Hoshi and Yamasaki \cite{Hoshi} 
classified algebraic tori of dimensions 4 and 5 up to stable rationality. 
Their classification is based on computing the flasque classes of algebraic 
tori in GAP. They showed that there are 3051 algebraic tori of deimension five 
which are stably rational and except for these the rest of them are not stably rational.
The rationality of these 3051 cases are not addressed in their work. Hoshi and 
Yamasaki's approach to prove stable rationality was different for algebraic tori with
an indecomposable character lattice. They showed that in rank 5, there are exactly 311 indecomposable 
$G$-lattices which are stably rational. More precisely they showed their 
stable rationality results by finding the maximal groups and proved stable rationality 
of their subgroups. The following table presents the maximal groups they found.

\begin{table}\label{tbl:18Max}
\centering
\begin{tabular}{llllll} 
Number & CARAT ID & $G$ & $\#G$\\ \hline
 1  & $(5,942,1)$ & ${\rm Imf}(5,1,1)$ & $3840$\\
2  & $(5,953,4)$ & $\mathrm{S}_6$ & $720$ \\
3  & $(5,726,4)$ & $C_2^4\rtimes \mathrm{S}_4$ & $384$  \\
4  & $(5,919,4)$ & $C_2\times \mathrm{S}_5$ & $240$ \\
5 & $(5,801,3)$ & $C_2\times (\mathrm{S}_3^2\rtimes C_2)$ & $144$ \\
6 & $(5,655,4)$ & $D_8^2\rtimes C_2$ & $128$ \\
7  & $(5,911,4)$ & $\mathrm{S}_5$ & $120$ \\
8 & $(5,946,2)$ & $\mathrm{S}_5$ & $120$ \\
9 & $(5,946,4)$ & $\mathrm{S}_5$ & $120$ \\
10 & $(5,947,2)$ & $\mathrm{S}_5$ ,\,\,\,\,\,\,\,\,\,\,\,\,\,\,\,\,\,\,\,\,\,\,\,\,\,\,\,\,\,\ & $120$ \\
11 & $(5,337,12)$ & $D_8\times \mathrm{S}_3$ & $48$ \\
12  & $(5,341,6)$ & $D_8\times \mathrm{S}_3$ & $48$ \\
13  & $(5,531,13)$ & $C_2\times \mathrm{S}_4$ & $48$ \\
14  & $(5,533,8)$ & $C_2\times \mathrm{S}_4$ & $48$ \\
15 & $(5,623,4)$ & $C_2\times \mathrm{S}_4$ & $48$\\
16   & $(5,245,12)$ & $C_2^2\times \mathrm{S}_3$ & $24$\\
17  & $(5,81,42)$ & $C_2\times D_8$ & $16$ \\
18  & $(5,81,48)$ & $C_2\times D_8$ & $16$ \\
\end{tabular}
\caption{The maximal $18$ groups 
in the  $311$ cases found by Hoshi and Yamasaki in \cite{Hoshi}.}
%\label{tbl:18Max}
\end{table}

In 2015 Lemire \cite{Nicole1}, proved that, except for possibly ten of the 
4 dimensional stably rational algebraic tori found by Hoshi and Yamasaki, 
all of them are rational. The rationality of the ten exceptional cases is 
still unknown. The author did not use any computer based arguments except 
for finding generating sets of groups and lattices of subgroups in GAP. We 
mainly follow the ideas from \cite{Nicole1} in an algorithmic way to make
these ideas computer based and then we use them to do experiments in dimension
five.

We present algorithms which may be applied to character lattices of algebraic 
tori, in order to investigate their rationality. These algorithms provide 
machinery to reduce the rationality problem in a specific dimension to lower dimensions
provided that the corresponding lattice is reducible.

Our contribution to the rationality problem is an experimental effort to 
test Voskeresenskii's conjecture in dimension five. Note that based on the results of 
\cite{Nicole1}, the conjecture is almost (since there are a few cases which are unresolved) 
true. In dimension five our methods fail to determine the rationality of some algebraic tori.
However, in any case of which our methods work, we got a positive answer.

We partition the set of five dimensional stably rational algebraic tori into the ones with a 
decomposable lattice and the ones with and indecomposable lattice. For simplicity we call an algebraic tors from the first family, decomposable and if it is from the second family we call it indecomposable. Both from theoretical and 
experimental points of view the indecomposable tori are more interesting. This is due to the
fact that the rationality of decomposable cases can be decided based on their decomposition
 components. This is basically done by forming components and looking up in the 
 known rationality tables. 

For indecomposable stably rational 5 dimensional algebraic tori,
we consider the ones where their groups are maximal among the 311 cases found by Hoshi and 
Yamasaki. These maximal groups are presented in \ref{tbl:18Max} and from now on we will call them
respectively $G_1$ to $G_{18}$. 

There are two 
main methods that we apply to $G_1$ to $G_{18}$ (and their subgroups), both of which were used in \cite{Nicole1}. The 
first method is reducing the rationality of a five dimensional torus to rationality 
in lower dimensions using a reducible component of corresponding lattices. More precisely, assuming $L_G$ is the corresponding lattice to a given indecomposable torus, we try to find a $G$-lattice, $M\subset
L_G$ so that we get the exact sequence 
$$0 \longrightarrow M \longrightarrow L_G \longrightarrow P \longrightarrow 0, $$
where $P$ is a permutation $G$-lattice (see Section \ref{sec:g-lattices}) for a definition).

The second method is to identify character lattices of indecomposable tori as lattices of which their rationality is known such as the root lattice of the root system $B_n$ or a sign permutation lattice.

\todo{give detail on results}

Definitions and useful facts about $G$-lattices are provided in Section \ref{sec:g-lattices}. Section 
\ref{sec:gap} gives brief information on computer algebra system GAP and relevant computational 
notations for groups based on GAP package, Carat. Section \ref{sec:ratres} is devoted to present 
theoretical known results on rationality problem for algebraic tori and rational families of algebraic 
tori. Our algorithms to find exact sequence of lattices are introduced in Section \ref{sec:redalg}. Finally Section \ref{sec:results} contains tables of results of our experiments.
 

\section{$G$-Lattices}\label{sec:g-lattices}
It is known that there is a duality between algebraic tori and \glat s. In order to be able to discuss the duality, we briefly introduce \glat s in this section. For a more detailed discussion see \cite[Chapters 1 and 2]{Lorenz}.\\ \\
A lattice is a free $\mathbb{Z}$-module of finite rank. As a $\Z$-module, it is isomorphic to $\mathbb{Z}^n$ for some $n$. If we have a group $G$, we can endow the lattice with an action of $G$. If $G$ acts on a lattice $L$ by automorphisms i.e. there exists $$ G \longrightarrow \mathrm{GL}(M) \cong \mathrm{GL}(n,\mathbb{Z}),$$ 
we say $L$ is a $G$-lattice. We can equivalently say that $\G$-lattices are free $\Z$-modules of finite rank which is also $\mathbb{Z}[G]$-modules, for $\mathbb{Z}[G]$ the integral group ring over $G$. A $G$-equivariant, $ \mathbb{Z}$-linear map between $G$-lattices is called a homomorphism of $G$-lattices.\\
\\For a given \glat \, $L$ we define $$ L^\G = \lbrace l \in L : g.l=l,  \,\, \,\,  \forall g \in \G \rbrace. $$

Now let $R$ be an arbitrary commutative (unital) ring. For any \glat  \, $L$ we can form the group algebra $R[L]$ and the action of $G$ on $L$ can be extended to an action on $R[L]$. The subalgebra of all $G$-invariant elements in $R[L]$ $$ R[L]^ \G = \lbrace l \in R[L] : g.l = l, \,\,\, \forall g \in \G \rbrace$$ is called the multiplicative invariant algebra. Studying this algebra is the subject of multiplicative invariant theory. 
\\
\\
The group algebra $R[L]$ of a \glat , is isomorphic to the Laurent polynomial algebra \linebreak $R[x_1^{\pm 1}, x_2^{\pm 1}, \ldots , x_n^{\pm 1}] $ and the lattice itself becomes a multiplicative subgroup of all monomials.
\\
\\
If $L$ and $L'$ are two \glat  s then $\mathrm{Hom}_{\Z}(L,L')$, the set of all $\Z$-linear maps from $L$ to $L'$, is a \glat \,\, with the following action $$ (g.f)(m) = g.(f(g^{-1}.m)).$$
For $L'=\Z$ (with trivial action of $G$), we get the \glat \,\,  $L^* = \mathrm{Hom}_{\Z}(L,\Z)$ which is called the dual lattice of $L$.\\
\\Suppose $L$ is a $G$-lattice and $H$ is a subgroup of $G$. $L$ can be considered as an $H$-lattice. The new $H$-lattice is called restricted and is denoted by $L\downarrow^G_H$.Assume $H$ is a finite index subgroup of $G$ and $L$ is an $H$-lattice. The $G$ module $\Z[H] \otimes_{\Z[G]} L$ is a $G$-lattice which is called the induced $G$-lattice and is denoted by $L\uparrow_H^G$. 

A \glat \,\, $L$ is called a permutation lattice if there exists a $\Z$ -basis $X$ for $L$ such that, $G$ acts as permutation group on $X$. In this case $L$ is sometimes denoted by $\Z[X]$.
We call a \glat \,$L$, decomposable if there are nontrivial \glat s $L_1$ and $ L_2$ such that $ L\cong L_1 \oplus L_2$. $L$ is called indecomposable if it is not decomposable.
If $L$ is decomposable then it is reducible, but the converse is not true for lattices.
A \glat \,$L$ is called reducible if $L$ has a nontrivial $G$-invariant subspace $M$ such that $L/M$ is torsion free. $L$ is called irreducible if it is not reducible. 

$L_G$ represents the corresponding $G$-lattice 
to a finite subgroup of $\mathrm{GL}(n,\Z)$, $G$, as defined in Definition 
\ref{Assumption}. When we say a group or a lattice is rational we mean their 
corresponding algebraic torus is rational. By a decomposable (matrix) group, 
we mean its corresponding lattice is decomposable. We say $G'$ is the dual 
group of $G$ if $G'$ is the corresponding group to the dual of $L_G$.  


\section{GAP: Carat and CrystCat}\label{sec:gap}
GAP \cite{GAP4} stands for Groups, Algorithms, Programming, and is a computer 
algebra system for computations in discrete algebra with emphasis on computations 
in group theory. GAP is an open source system which is accessible directly or 
in SAGE \cite{sagemath}. GAP provides various packages for computations in matrix 
groups and representation theory. For our purposes we need Carat and CrystCat packages of GAP.\\
\\
The GAP package Carat provides functions of the stand-alone programs of $CARAT$, 
which is a package for the computations related to crystallographic groups. Carat 
contains the catalog of all conjugacy classes of finite subgroups of $\mathrm{GL}(n,\Z)$ 
for $n$ up to six. More precisely the Carat package gives access to all 
$\Q$-classes and $\Z$-classes and maximal classes over $\Z$.  
\begin{remark}
The $\Q$-classes are conjugacy classes over $\Q$. We note that some $\Z$-classes 
may belong to the same conjugacy class over the rationals.
\end{remark}

It is worth mentioning that Carat contains information about crystallographic groups 
which we will not use. The CrystCat Package in GAP also provides a catalog of 
crystallographic groups up to dimension 4. The catalog mostly covers the data in 
\cite{Crystallography}. CrystCat and Carat are complement of each other. \\
\\
The GAP ID, $(n,m,l,k)$ of a finite subgroup $G$ of $\mathrm{GL}(n,\Z)$ means that 
$G$ is of rank $n$ and belongs to $k$-th $\Z$ class of the $l$-th $\Q$-class of the 
$m$-th crystal system. This works for $2\leq n \leq 4$. Hoshi and Yamasaki wrote 
a GAP code using the Carat package to have easy access to the $j$-th $\Z$-class 
of the $i$-th $\Q$-class group of rank $n$. They called this Carat ID. The GAP 
scripts written by Hoshi and Yamasaki are available from
\begin{center}
http://www.math.h.kyoto-u.ac.jp/~yamasaki/Algorithm/
\end{center}
The algorithms introduced in this paper are implemented in GAP (needs some 
functions from the codes written by Hoshi and Yamasaki) and the code is available from
\begin{center}
https://github.com/armin-jamshidpey/Algebraic-Tori
\end{center}
Since the actions of matrix groups in GAP are considered from right, throughout 
this chapter we work with row vectors instead of columns. One may also use the 
columns by considering the dual groups.\\
\\
We call $\G \leq \mathrm{GL}(n,\Z)$ irreducible (resp. indecomposable), if the corresponding 
lattice to $G$ be irreducible (resp. indecomposable).

\section{Well-Known Rationality Results}\label{sec:ratres}
In this section we present some important results about rationality problem for 
algebraic tori. Before presenting the results we need the following definition 
to avoid repeating the same assumptions.
\begin{definition}\label{Assumption}
If $G$ is a finite subgroup of $\mathrm{GL}(n,\Z)$, then the corresponding lattice 
to $G$ which is denoted by $L_G$ is the rank $n$ lattice generated by the standard 
basis, i.e. $L_G = \langle e_i : i  = 1, \ldots, n \rangle_\Z$ where $(e_i)_j = \delta_{ij}$. 
The action of $G$ on $L_G$ is given by multiplication from right on the $e_i$'s.
 Moreover, if $G \cong \mathrm{Gal}(K/F)$ for some finite Galois extension $K/F$ 
 then $K[L_G] \cong K[x^{\pm 1}_1, \ldots , x^{\pm 1}_n]$, that is the Laurent 
 polynomial ring, and $K(L_G)$ which is the quotient field of $K[L_G]$ is isomorphic 
 to $K(x_1, \ldots , x_n)$ ($x_i$'s are algebraically independent over $K$) are equipped 
 with an action of $G$ as
\begin{itemize}
\item $G$ acts as Galois group on $K$ 
\item $\forall g \in G, \,\,\, g(x_i) = \prod_{j=1}^{j=n} x_j^{a_{ij}}$  where $a_{ij}$'s 
are given by $g(e_i) = \sum_{j=1}^{j=n} a_{ij}e_j$.
\end{itemize}
Also $T_G$ is the corresponding algebraic torus to $L_G$ i.e. $T_G$ is an algebraic 
torus defined on $F$ which splits over $K$, with character lattice $L_G$. 
\end{definition}
\noindent
By the duality explained before, $K(L_G)^G$ is the rational function field of $T_G$. 
From now on we work with finite subgroups of $\mathrm{GL}(n,\Z)$ (up to conjugacy) and 
when we consider their corresponding lattice (or algebraic torus), $L_G$ ($T_G$), we 
mean the lattice (algebraic torus) defined in Definition \ref{Assumption}. 

The first family of of rational algebraic tori is given by the following theorem which 
shows that any algebraic torus with a permutation character group is rational.

\begin{theorem}\cite{Speiser}\label{thm:no-name lemma}
(no-name lemma) Let $M$ be a permutation \glat \,\,and $K$ be a \G-field. Then $K(M)^\G$ 
is rational over $K^\G$. 
\end{theorem}

Note that if $\G = \mathrm{Gal} (K/F)$, then $K^\G = F$. For a constructive proof see 
\cite{JamLemSch2019}, also a more general version of the no-name lemma see \cite{Domokos}. The following 
theorem generalizes the previous result to sign permutation lattices. A $G$-lattice is called 
sign permutation if it has a basis $X$ which is permuted up to sign changes under the action of $G$.
\begin{theorem}\cite[Proposition 9.5.1]{Lorenz}\label{thm:signpermrat}
Assume $L$ is a sign permutation (permutation up to sign changes) $G$-lattice and $K$ is a $G$-field. Then $K(M)^G$ 
is rational over $K^G$.
\end{theorem}
The rationality problem for algebraic tori of dimension one was concrete. For 
dimension two Voskresenskii used a geometric method to prove the below result. 
\begin{theorem}
\cite{Vos67} Any 2 dimensional algebraic torus over k, is k-rational.
\end{theorem} 
The following two theorems classifies algebraic tori of dimension 4 and  5 up to stable 
rationality. In \cite{Hoshi} the authors gave a complete classification of mentioned tori, 
however they did not say anything about rationality of tori of dimension 4 and 5. The 
main idea of their work was to investigate the last 3 results above, by means of computer 
algebra system, GAP. 
\begin{theorem}\cite[Theorem 1.9]{Hoshi}
Let $K/F$ be a finite Galois extension with Galois Group $\G \leqslant \mathrm{GL}(4, \Z)$. 
Assume $G$ acts on $L = K(x_1,x_2,x_3,x_4)$ as above. (For tables of the below subgroups 
see \cite[Page 4]{Hoshi})
\\ 
(i) $L^\G$ is stably $F$-rational if $G$ is (up to conjugacy) one of a list of 487 
subgroups of $\mathrm{GL}(4,\Z)$.
\\ 
(ii)  $L^\G$ is not stably but retract $F$-rational if $G$ is (up to conjugacy) one 
of a list of 7 subgroups of $\mathrm{GL}(4,\Z)$.
\\
(iii) $L^\G$ is not retract $F$-rational if $G$ is (up to conjugacy) one of a list of 
216 subgroups of $\mathrm{GL}(4,\Z)$.
 \end{theorem}
In 2015, Lemire showed that except for possibly ten, all stably rational groups found 
by Hoshi and Yamasaki are rational (see \cite{Nicole1}).
\begin{theorem}\label{thm:Hoshi}\cite[Theorem 1.12]{Hoshi}
Let $K/F$ be a finite Galois extension with Galois Group $\G  \leqslant \mathrm{GL}(5, \Z)$. 
Assume G acts on $L = K(x_1,x_2,x_3,x_4,x_5)$ as above. (for tables of below subgroups see 
\cite[Pages 134-144]{Hoshi})
\\ 
(i) $L^\G$ is stably $F$-rational if $G$ is (up to conjugacy) one of a list 3051 subgroups 
of $\mathrm{GL}(5,\Z)$.
\\ 
(ii)  $L^\G$ is not stably but retract $F$-rational if $G$ is (up to conjugacy) one of a list 
25 subgroups of $\mathrm{GL}(5,\Z)$.
\\
(iii) $L^\G$ is not retract $F$-rational if $G$ is (up to conjugacy) one of a list of 3003 
subgroups of $\mathrm{GL}(5,\Z)$.
 \end{theorem}

There are examples of varieties which are stably rational but not rational. So in general 
being stably rational is not the same as being rational. However, there is a conjecture 
about the equivalence of stable rationality and rationality for algebraic tori.\\ \\
 \textbf{Conjecture}. \cite[Section 2.6.1]{Voskresenskii} Any stably rational algebraic 
 torus is rational. 
 
According to Theorem \ref{thm:Hoshi} we know all stably rational algebraic tori of dimension 5. 
However, the theorem does not say anything about rationality of those tori. An interesting 
problem is to find all rational tori between 3051 mentioned tori in the theorem.
 
Later on, we investigate the rationality of stably rational algebraic tori 
of dimension 5.  We will try to reduce their rationality to the rationality of some 
well understood algebraic torus. Here we present some families of 
algebraic tori which are rational, so that we can relate our algebraic tori to one 
of these families. It is already mentioned that every $n$ dimensional algebraic torus 
has a corresponding finite subgroup of $\mathrm{GL}(n,\Z)$. In order to study the 
rationality of algebraic tori, we study its corresponding group. We would rather to 
consider maximal groups and prove rationality for their subgroups, instead of proving 
it case by case.  

Let $L$ be a $G$-lattice for $G\leq \mathrm{GL}(n,\Z)$. Following \cite{Nicole1}, if all algebraic tori with character lattice $L\downarrow_H^G$ and splitting group $H$ are rational, for any 
subgroup $H \leq G$, then we call $L$ hereditarily rational. 
Moreover, if $T$ is an algebraic torus and $L$ is its corresponding lattice, then $T$ is called 
hereditarily rational if $L$ is hereditarily rational.

By Theorem \ref{thm:no-name lemma}, a quasi-split torus is rational. For a permutation 
$G$-lattice $L$ and any subgroup $H\leq G$, since $L\downarrow_H^G$ is a permutation 
lattice, the corresponding torus to $L\downarrow_H^G$ is rational. In other words a 
quasi split torus is hereditarily rational. Simillarly by Theorem \ref{thm:signpermrat} 
and above argument for (a sign permutation lattice), we conclude that any algebraic 
torus with a sign permutation character lattice is hereditarily rational. 
In particular this is true for an algebraic torus with character lattice the root 
lattice $\Z B_n$ as an $W(B_n)$-lattice. It is also known that any rank $n$ sign 
permutation lattice, is isomorphic to the restriction of $\Z B_n$ to a subgroup 
of $W(B_n)$.  

The following theorem provides an important tool for our algorithm to determine rationality of a
reducible torus. 
 
\begin{theorem}\cite[Proposition 1.6]{Lenstra}\label{thm:lenstra}
Suppose $P$ is a permutation $G$-lattice and $G$ is the Galois group of a finite 
Galois extension, $K/F$.  If 
$$0 \longrightarrow M \longrightarrow L \longrightarrow P \longrightarrow 0 $$ is 
an exact sequence of $G$-lattices, then $K(L)^G$ is rational over $K(M)^G$.
\end{theorem}

An important corollary of the above theorem will be used frequently in chapter 3, 
in order to prove the rationality of algebraic tori.
\begin{corollary}\label{permcoker}
Suppose $P$ is a permutation $G$-lattice and $G$ is the Galois group of a finite 
Galois extension, $K/F$.  If $$0 \longrightarrow M \longrightarrow L \longrightarrow P \longrightarrow 0 $$ 
is an exact sequence of $G$-lattices and $K(M)^G$ is rational over $F$, then $K(L)^G$ is rational over $F$.
\end{corollary}

In \cite[Section 2.4.8]{Voskresenskii} the author has shown that any algebraic 
torus with an augmentation ideal lattice is hereditarily rational. More precisely 
let $T$ be an algebraic torus defined over $F$ and splits over $K$ and $G = \mathrm{Gal}(K/F)$. 
Assume the character lattice of $T$ is $I_X$ (the kernel of the augmentation map), 
and $\Z [X]$ is a $G$-permuattion lattice, where 
\begin{equation}\label{Chevalley}
0 \longrightarrow I_X \longrightarrow \Z [X] \overset{\varepsilon} \longrightarrow  \Z \longrightarrow 0
\end{equation}
is an exact sequence and $\varepsilon : \Z [X]  \rightarrow \Z $, $x \rightarrow 1$ 
is the augmentation map. The exact sequence (\ref{Chevalley}) corresponds to the exact 
sequence of $F$ algebraic tori
$$0 \longrightarrow \mathbb{G}_m \longrightarrow R_{K_1/F}(\mathbb{G}_m) \times \cdots \times R_{K_t/F}(\mathbb{G}_m) \longrightarrow  T \longrightarrow 0$$
where $K_i/F$ (for $i = 1, \ldots ,t $) are intermediate fields of $K/F$ and $K/K_i$ is 
Galois. Now $T = \prod^t_{i = 1}R_{K_i/F}(\mathbb{G}_m)/\mathbb{G}_m$ and is rational. 
We note that for any subgroup $H$ of $G$, ${I_X}\downarrow^G_H$ is also an augmentation 
ideal. Hence an algebraic tori with augmentation ideal character lattice is hereditarily 
rational.

It is worth mentioning that passing to dual lattices in \ref{Chevalley} we get 
$$0 \longrightarrow \Z \longrightarrow \Z [X] \overset{\varepsilon} \longrightarrow  J_{X} \longrightarrow 0$$
where $J_{X}= I^{\ast}_X$ is called Chevalley module. The corresponding algebraic 
torus to $J_X$ has interesting properties and is called norm one torus. Chevalley 
was the first one who discovered that norm one torus is not necessarily rational. 

The following lemma was used in \cite{Nicole1} to show that a given $G$-lattice 
is isomorphic to $J_{G/H}$.
\begin{lemma}\cite[Remark 4.1]{Nicole1}
Let $L$ be a $G$-lattice. If there exist $x\in L$ such that,
\begin{itemize}
\item $\langle G.x \rangle_{\Z} = L$
\item $\mathrm{Stab}_G(x) = H$
\item $\sum_{g \in G}gx = 0$,
\end{itemize}
then $L \cong J_{G/H}$.
\end{lemma}

\section{Reduction Algorithms}\label{sec:redalg}
Assume $$0 \rightarrow M  \rightarrow L_G \rightarrow N \rightarrow 0$$ is a short 
exact sequence of $G$-lattices such that $N$ is a permutation projective $G$-lattice. 
If $K/F$ is a finite Galois extension with $G \cong \mathrm{Gal}(K/F)$, then by 
Theorem (\ref{thm:lenstra}), $K(L_G)^G$ is rational over $K(M)^G$. Thus, rationality 
of $K(M)^G$ over $F$ implies rationality of $K(L_G)^G$ over $F$.

Suppose $L_G$ is an indecomposable $G$-lattice. In this section, we present methods 
to examine the possibility of existence of such a short exact sequence for $L_G$, with
 $N$ a permutation $G$-lattice.
 
Although sign permutation lattices are not permutation projective, constructing a 
short exact sequence of $G$-lattices $$0 \rightarrow M  \rightarrow L_G \rightarrow N \rightarrow 0$$ 
where $N$ is a rank one sign permutation $G$-lattice might help to determine 
rationality of the associated algebraic torus to $L_G$. Note that existence of 
such a sequence does not directly imply rationality. However, under some conditions 
the rationality may be concluded. 

The goal of this section is to provide tools to get exact sequences mentioned above 
for a given indecomposable $G$-lattice. The idea behind all of the methods is a simple 
fact which we explain briefly here. 

A lattice $L_G$, is reducible as a $G$-lattice if and only if $\Q L_G = L_G \otimes_{\Z} \Q$ 
has a proper $\Q[G]$-submodule $W$ of dimension $0 < m < n$. Let $L_G$ be a $G$-lattice of 
rank $n$ and $W$ is an $m$ dimensional proper $\Q[G]$-submodule of $\Q L_G$. Then $L_G \cap W$ 
is a sublattice of $L_G$ of rank $m$ such that $\Q (L_G \cap W) = W$. Then
$$0 \rightarrow L_G \cap W  \rightarrow L_G \rightarrow L_G/(L_G \cap W) \rightarrow 0$$
 is a short exact sequence of $G$-lattices. Note that this implies in particular that $L_G/(L_G \cap W)$ 
 is torsion free so that a $\Z$-basis of $L_G \cap W$ can be extended to a $\Z$-basis of $L_G$.
 
In the next paragraphs we are specifically looking for an $n-1$ dimensional proper 
$\Q[G]$-submodule of $\Q L_G$.

If we start with the dual lattice $L^*_G$, and we are able to find a rank 1 permutation 
sublattice of $L^*_G$, we get $$0 \rightarrow \Z  \rightarrow L^*_G \rightarrow M \rightarrow 0,$$ 
where $M = L^*_G/\Z$ is of rank $n-1$. Then by dualizing the sequence we have 
$$0 \rightarrow M^*  \rightarrow L_G \rightarrow \Z \rightarrow 0$$ as desired.

Now, we explain how to find a permutation rank one sublattice of $L^*_G$. In order 
to get a one dimensional $\Q[G]$-submodule of $\Q L^*_G$, we use the eigenspaces of 
the transposes of a generating set of $G$. Let  $\lbrace \sigma_1, \ldots, \sigma_m\rbrace$ 
be the transposes of a generating set of $G$ and let  $G^* = \langle \sigma_1, \ldots, \sigma_m\rangle$. 
Suppose $E_{1,\sigma_i}$ is the left nullspace of $\sigma_i-I$ over $\Q$. 
We define $$E_1 = E_{1,\sigma_1}\cap \cdots \cap E_{1,\sigma_n}. $$
Note that $G^*$ acts trivially on $E_1$. If $E_1\neq {0}$ then we can choose a 
nonzero vector $u \in E_1$. Let $u = (\frac{a_1}{b_1}, \frac{a_2}{b_2}, \ldots , \frac{a_n}{b_n})\in E_1$ 
such that $\gcd(a_i, b_i) = 1$. If $m = \mathrm{lcm}(b_1, \ldots, b_n)$ then 
$m u = (a'_1, \ldots, a'_n) \in\Z^n$. If $\gcd (a'_1, \ldots, a'_n) = d$ then 
$v =\frac{m}{d} u \in L_G \cap E_1$ and the $\gcd$ of its entries is 1. 

As a consequence, we can extend $\lbrace v \rbrace$ to a $\Z$-basis of $L^*_G$. 
A general algorithm to do this extension is given by Magliveras et al in 
\cite{LatticeBase}; GAP also has a function which does the job. In most of the 
cases that we will see in the next section, $v$ had a $\pm 1$ as an entry, which makes 
the basis extension so simple: if $v_j$ is $\pm 1$ then 
$\lbrace e_1, \ldots , e_{j-1}, e_{j+1}, \ldots, e_n , v \rbrace$ forms a $\Z$-basis for $L^*_G$.

Since it is possible to extend $v$ to a basis for the lattice $L^*_G$, there exists a change of basis matrix $T$ in $\mathrm{GL}(n,\Z)$ such that
$$T \sigma_i T^{-1} = \left[ \begin{array}{c|c}
\delta_i & \ast\\
\hline 0 & 1
\end{array} \right]
$$ 
for some $\delta_i \in \mathrm{GL}(n-1,\Z).$ Since we consider the finite subgroups of $\mathrm{GL}(n,\Z)$ up to conjugacy, we can work with $G' = TG^*T^{-1}$. By considering the first $n-1$ vectors of the standard basis of the $G'$-lattice $L_{G'}$ (which is isomorphic to $L^*_G$), can form the $G'$-lattice $M$ such that $L_{G'}/\Z = M$ and we get 
$$0 \longrightarrow \Z \longrightarrow L_{G'} \longrightarrow M \longrightarrow 0.$$
By dualizing the sequence we get  $$0 \longrightarrow M^* \longrightarrow L^*_{G'} \longrightarrow \Z \longrightarrow 0.$$
Note that $L^*_{G'}$ is isomorphic to $L_G$.

The explained method above is presented as an algorithm here.
\begin{algorithm}[H]
	\caption{Fixed Point Algorithm}
	\label{alg:Fixed Point}
	\begin{algorithmic}[1]
		\REQUIRE A finite subgroup $G$ of $\mathrm{GL}(n,\Z)$, given by its generators $\lbrace \sigma_1, \ldots , \sigma_m \rbrace$.
		\ENSURE A matrix $T \in \mathrm{GL}(n,\Z)$ such that $T \sigma^t_i T^{-1}= \left[ \begin{array}{c|c}
		\delta_i & \ast \\
		\hline 
		0 &1
		\end{array} \right]
		$ where $\delta_i \in \mathrm{GL}(n-1,\Z)$, and sublattices $M, N$ \\
		\hspace{0.8cm}such that $0\longrightarrow N \longrightarrow L^*_G \longrightarrow M \longrightarrow 0$ 
		is an exact sequence of lattices.
		\bigskip
		\STATE $E \gets \left[ \begin{array}{c|c|c|c}
		\sigma^t_1-I & \sigma^t_2 -I & \cdots & \sigma^t_n -I
		\end{array} \right]$
	    \STATE $W \gets \mathrm{LeftNullspace}(E)$
	    \STATE if $W $ is not zero then\\
	    \hspace{0.5in} choose a nonzero $v \in W$\\
	    \hspace{0.5 in} if $v \notin \Z^n$ \\
	    \hspace{1in} find $c \in \Z$ s.t $cv \in \Z^n$ and $\gcd(cv) = 1$\\
	    \hspace{1in} $v \gets cv$\\
	    \hspace{0.5 in} end if\\
	    \hspace{0.5in} apply the algorithm in \cite{LatticeBase} to extend $v$ to a basis $B = \lbrace \beta_1, \ldots, \beta_{n-1},v\rbrace$ for $L_G$\\
	    \hspace{0.5in} $T \gets \begin{bmatrix}
	    \beta_1 &  \cdots &  \beta_{n-1}& v
	    \end{bmatrix}^t$\\
		\hspace{0.5in} $N \gets \Z v$\\
		\hspace{0.5in} $M \gets L/N$\\
		\hspace{0.5in} \textbf{return} $M,N, T$\\
	  \hspace{0.5in} end if\\
	  else\\
	     \hspace{0.5in}  \textbf{return} fail\\
	     end if
	 	\end{algorithmic}
\end{algorithm}
\begin{remark}
If the algorithm returns a matrix $T$ then for $\sigma \in \lbrace \sigma^t_1, \ldots , \sigma^t_m \rbrace$, 
$$T \sigma T^{-1} = \sigma'$$ where $$\sigma' = \left[ \begin{array}{c|c}
\delta & \ast \\
\hline
0 &1 
\end{array}\right]. 
$$
for some $\delta \in \mathrm{GL}(n-1, \Z)$. More precisely $$T \sigma = \sigma'T$$ and the last row of $T \sigma$ is nothing but $v \sigma = v$. This implies that the last row of $\sigma' = [0 \,\, \ldots \,\, 0 \,\, 1]$.
\end{remark}
\begin{example}
Let $G\leq \mathrm{GL}(4,\Z)$ be generated by 
$$ \left[ \begin {array}{cccc} 0&-1&0&0\\ -1&-1&-1&-1
\\ 1&1&1&0\\ 1&1&0&1\end {array}
 \right] 
\tand
\left[ \begin {array}{cccc} 0&1&0&0\\ -1&0&0&0
\\ 0&-1&0&-1\\ 0&-1&-1&0
\end {array} \right].
$$
The transposes are 
$$
 \sigma = \left[ \begin {array}{cccc} 0&-1&1&1\\ -1&-1&1&1
\\ 0&-1&1&0\\ 0&-1&0&1\end {array}
 \right]
 \tand 
\tau =  \left[ \begin {array}{cccc} 0&-1&0&0\\ 1&0&-1&-1
\\ 0&0&0&-1\\ 0&0&-1&0\end {array}
 \right].
$$ 
Then the $E_{1,\sigma}$ is the left nullspace of $\sigma - I_4$.
One can verify that 
\begin{displaymath}
\lbrace \begin{bmatrix}
1& -1 &1 &0
\end{bmatrix}, \begin{bmatrix}
1& -1 &0 &1
\end{bmatrix}
\rbrace
\end{displaymath}
is a basis for $E_{1,\sigma}$. Similarly $E_{1,\tau}$ is the left nullspace of $\tau - I_4$
 which is generated by $\begin{bmatrix}
0& 0 &-1 &1
\end{bmatrix}$.\\
\\It is not hard to see $\begin{bmatrix}
0& 0 &-1 &1
\end{bmatrix}\in E_{1,\sigma}$. Thus $$E_1 = E_{1,\sigma} \cap E_{1,\tau} = \langle \begin{bmatrix}
0& 0 &-1 &1
\end{bmatrix} \rangle$$
and $ \begin{bmatrix}
0& 0 &-1 &1
\end{bmatrix}\in L$. Now $$\lbrace  \begin{bmatrix}
1& 0 &0 &0
\end{bmatrix} ,  \begin{bmatrix}
0& 1 &0 &0
\end{bmatrix},  \begin{bmatrix}
0& 0 &1 &0
\end{bmatrix},  \begin{bmatrix}
0& 0 &-1 &1
\end{bmatrix} \rbrace$$ forms a $\Z$-basis for $L$. The change of basis matrix $T$ is 
$$
\begin{bmatrix}
1 & 0 &0 &0\\
0 & 1 &0 &0\\
0 & 0 &1 &0\\
0 & 0 &-1 &1
\end{bmatrix}
$$ Hence 
$$
T \sigma T^{-1}= \left[ \begin {array}{ccc|c} 0&-1&2&1\\ -1&-1&2&1
\\ 0&-1&1&0\\ \hline 0&0&0&1\end {array}
 \right], \,\,\,
T\tau T^{-1}=  \left[ \begin {array}{ccc|c} 0&-1&0&0\\ 1&0&-2&-1
\\ 0&0&-1&-1\\ \hline 0&0&0&1\end {array}
 \right] 
$$
Now by dualizing we get 
$$
 \left[ \begin {array}{ccc|c} 0&-1&0&0\\ -1&-1&-1&0
\\ 2&2&1&0\\ \hline 1&1&0&1\end {array}
 \right]
 \tand 
 \left[ \begin {array}{ccc|c} 0&1&0&0\\ -1&0&0&0
\\ 0&-2&-1&0\\ \hline 0&-1&-1&1
\end {array} \right] 
$$ 
as a generating set for a conjugate of $G$. By defining $M\subset L_G$ generated by $e_1, e_2$ and $e_3$ the following exact sequence will be obtained
$$
0 \longrightarrow M \longrightarrow L_G \longrightarrow \Z \longrightarrow 0
$$ 
\end{example}
We call the above process the fixed point algorithm. One can generalize it as follows. Assume $E_{\lambda,\sigma}$ be the left kernel of $\sigma-\lambda I$ over the  rationals. Now define $E_{\pm 1,\sigma}$ to be  the set $\lbrace E_{1,\sigma}, E_{-1,\sigma}\rbrace$. Assume $E_{\pm 1, G}$ is the Cartesian product of $E_{\pm1, \sigma}$ for all $\sigma \in G^*$, i.e. 
$$E_{\pm 1,G} = E_{\pm 1, \sigma_1} \times \cdots \times E_{\pm 1, \sigma_n}.$$ 
If there exist $A\in E_{\pm 1,  G}$ such that $W = \bigcap\limits_{B \in A} B \neq 0$, then we can find a nonzero $v \in L^*_G \cap W$. As we have seen in the fixed point algorithm, we can extend a multiple of $v$ to a $\Z$-basis for $L^*_G$. Then we can get a change of basis matrix $T$, such that 
$$T \sigma_i T^{-1} = \left[ \begin{array}{c|c}
\delta_i & \ast\\
\hline
0 & \pm 1
\end{array}\right]
$$ 
for some $\delta_i \in \mathrm{GL}(n-1,\Z).$ Thus by a similar argument we can use the new representative of the conjugacy class of $G^*$ to form an equivalent lattice and similarly by choosing the first $n-1$ elements of the standard basis of $L_{G'}$ we can produce 
$$0 \longrightarrow \Z^- \longrightarrow L_{G'} \longrightarrow M \longrightarrow 0.$$
By dualizing the sequence we get 
$$0 \longrightarrow M^* \longrightarrow L^*_{G'} \longrightarrow \Z^- \longrightarrow 0.$$
Again note that $L^*_{G'}$ is isomorphic to $L_G$.\\
\\
This process will be called the sign fixed point algorithm and it is presented as an algorithm here. 
\begin{algorithm}[H]
	\caption{Sign Fixed Point Algorithm}
	\label{alg:Sign Fixed Point}
	\begin{algorithmic}[1]
		\REQUIRE A finite subgroup $G$ of $\mathrm{GL}(n,\Z)$, given by its generators $\lbrace \sigma_1, \ldots , \sigma_m \rbrace$.
		\ENSURE A matrix $T \in \mathrm{GL}(n,\Z)$ such that $T \sigma^t_i T^{-1}= \left[ \begin{array}{c|c}
		\delta_i & \ast \\
		\hline 
		0 &1
		\end{array} \right]
		$ where $\delta_i \in \mathrm{GL}(n-1,\Z)$, and sublattices $M, N$\\ 
		\hspace{.8 cm}such that $0\longrightarrow N \longrightarrow L^*_G \longrightarrow M \longrightarrow 0$ is an exact sequence of lattices.
		\bigskip
		\STATE	 for g in $\lbrace	\sigma^t_1, \ldots , \sigma^t_m \rbrace$ do\\
		\hspace{0.5 in} $E_g \gets $ the set of left nullspaces of $g \pm I$ over $\Q$\\
		end do
		\STATE $E \gets E_{\sigma^t_1} \times E_{\sigma^t_2} \times \cdots \times E_{\sigma^t_m}$
		\STATE $W \gets 0$
		\STATE while $W = 0$ and $E \neq \emptyset$ do\\
		\hspace{0.5in} $A$ $\gets$ a random element of $E$ \\
	    \hspace{0.5in}$W \gets \bigcap\limits_{a\in A}a$\\
	      \hspace{0.5in}$E \gets E \setminus A$\\
	    end do
	    \STATE if $W $ is not zero then\\
	    \hspace{0.5in} choose a nonzero $v \in W$\\
	    \hspace{0.5 in} if $v \notin \Z^n$ \\
	    \hspace{1in} find $c \in \Z$ s.t $cv \in \Z^n$ and $\gcd(cv) = 1$\\
	    \hspace{1in} $v \gets cv$\\
	    \hspace{0.5 in} end if\\
	    \hspace{0.5in} apply the algorithm in \cite{LatticeBase} to extend $v$ to get a basis $B = \lbrace \beta_1, \ldots, \beta_{n-1},v\rbrace$ for $L$\\
	    \hspace{0.5in} $T \gets \begin{bmatrix}
	    \beta_1 &  \cdots &  \beta_{n-1}& v
	    \end{bmatrix}^t$\\
		\hspace{0.5in} $N \gets \Z v$\\
		\hspace{0.5in} $M \gets L/N$\\
		\hspace{0.5in} \textbf{return} $M,N,T$\\
	  \hspace{0.5in} end if\\
	  else\\
	     \hspace{0.5in}  \textbf{return} fail\\
	     end if
	 	\end{algorithmic}
\end{algorithm}

\begin{example}
Let $G\leq \mathrm{GL}(4,\Z)$ be generated by
$$
\left[ \begin {array}{cccc} -1&1&0&1\\ 0&0&0&1
\\ -1&0&1&1\\ 0&1&0&0\end {array}
 \right] 
\tand
\left[ \begin {array}{cccc} 0&1&0&0\\ 0&0&1&0
\\ 0&0&0&-1\\ -1&1&0&1\end {array}
 \right].
$$ 
The transposes are
$$
\sigma = \left[ \begin {array}{cccc} -1&0&-1&0\\ 1&0&0&1
\\ 0&0&1&0\\ 1&1&1&0\end {array}
 \right] 
\tand 
\tau =  \left[ \begin {array}{cccc} 0&0&0&-1\\ 1&0&0&1
\\ 0&1&0&0\\ 0&0&-1&1\end {array}
 \right].
$$ 
By computing the left nullspaces ($\sigma \pm I_4$ and $\tau \pm I_4$) we get 
$$E_{1,\sigma} =\langle [ 0, 0, 1, 0 ], [ 1, 1, 0, 1 ] \rangle, \,\,\, 
E_{-1,\sigma} =\langle [ [ 2, 0, 1, 0 ], [ 1, -1, 0, 1 ] ] \rangle $$
$$E_{1,\tau} =\langle [ -1, -1, -1, 1 ]  \rangle, \,\,\,
E_{-1,\tau} =\langle [ 1, -1, 1, 1 ]\rangle. $$
So 
$$ E_{1,\sigma} \cap E_{1,\tau}= {0}, \,\,\,
 E_{1,\sigma} \cap E_{-1,\tau}= {0} $$
$$ E_{-1,\sigma} \cap E_{1,\tau}= \langle [ 1, 1, 1, -1 ] \rangle, \,\,\,
 E_{-1,\sigma} \cap E_{-1,\tau}= {0}. $$
Let $W = E_{-1,\sigma} \cap E_{1,\tau}.$ So $[ 1, 1, 1, -1 ] \in L \cap W$ is extendable to a basis for $L$ by vectors 
$$[ 1, 0, 0, 0 ] \, , \, [ 0,1,0,0 ] \tand [ 0,0,1,0] $$
and the corresponding transformation is 
$$
T = \begin{bmatrix}
1 & 0 & 0 & 0\\
0 & 1 & 0 & 0\\
0 & 0 & 1 & 0\\
1 & 1 & 1 & -1
\end{bmatrix}
$$
$$
\sigma' = T\sigma T^{-1} = \left[ \begin {array}{cccc} 0&1&-1&-1\\ 1&0&1&0
\\ 0&0&1&1\\ 0&0&0&-1\end {array}
 \right] 
, \,\,\,
\tau' = T \tau T^{-1} =\left[ \begin {array}{cccc} 0&0&-1&0\\ 1&0&1&0
\\ 1&1&0&-1\\ 0&0&0&1\end {array}
 \right].
$$
Now by dualizing we get 
$$
 \left[ \begin {array}{ccc|c} 0&1&0&0\\ 1&0&0&0
\\ -1&1&1&0\\ \hline -1&0&1&-1
\end {array} \right]
 \tand 
 \left[ \begin {array}{ccc|c} 0&1&1&0\\ 0&0&1&0
\\ -1&1&0&0\\ \hline 0&0&-1&1\end {array}
 \right] 
$$ 
as a generating set for a conjugate of $G$. By defining $M\subset L_G$ generated by $e_1, e_2$ and $e_3$ the following exact sequence will be obtained
$$0 \longrightarrow M \longrightarrow L_G \longrightarrow \Z^- \longrightarrow 0.$$
\end{example}

In general to get a $\Q[G]$-submodule $W$ of $\Q L$, where $L = L_G$ is a $G$-lattice, one can use the decomposition of $\Q L$ (provided that it is decomposable). It is not always easy to find the decomposition of a $\Q[G]$-module. The most well-known tool for module decomposition is the meataxe algorithm. The algorithm was first introduced by Parker in \cite{Parker} in order to check irreducibility of a finite dimensional module over a finite field and finding explicit submodules in case of reducibility. Later on Parker extended the idea of the meataxe algorithm to characteristic zero (see \cite{Parker2}). His algorithm can be used to decompose an integral representation of a finite group. In \cite{Plesken2}, the authors provided machinery which enables us to decompose $\Q[G]$-modules up to dimension 200. So there are algorithms which give the decomposition over $\Q$. We invite the reader to see \cite{Lux} and \cite{Holt} for more details. 

Assume $G \leq \mathrm{GL}(n,\Z)$ is finite and $\Q L = L \otimes_{\Z} \Q$ is the $\Q$-vector corresponding space to $L$. If $\Q L$ is a decomposable $\Q[G]$-module, then there exists a change of basis matrix such that generators of the $\Q$-class of $G$ can be written as block diagonal matrices
$$
T \sigma_i T^{-1} =  \left[ \begin{array}{c|c}
\delta_i & 0 \\
\hline
0 & \gamma_i
\end{array} \right]
$$
where $\delta_i	\in \mathrm{GL}(m,\Q)$ and $ \gamma_i \in \mathrm{GL}(m',\Q)$ for some $m, m' \in \Z$. Let $\lbrace e_1, \ldots, e_m, e_{m+1}, \ldots , e_{m+m'}\rbrace$ be the standard basis for $\Q L$. The $\Q M$ and $\Q N$ generated respectively by $\lbrace e_1, \ldots, e_m \rbrace$ and 
$\lbrace e_{m+1}, \ldots , e_{m+m'} \rbrace$ are invariant (set wise) under the action of $G$ and $\Q L = \Q M \oplus \Q N$. Now $T^{-1}(\Q M)$ is a $G$-stable subspace and $$M = L \cap T^{-1}(\Q M) $$ is $G$ stable. Then we get 
$$0 \longrightarrow M \longrightarrow L \longrightarrow L/M \longrightarrow 0$$
as an exact sequence of lattices.

The above idea can be turned into an algorithm. In order to do so, one need to compute the decomposition of $\Q L$ (meataxe or any other algorithm can be applied). If in the previous step the change of basis, namely $T$, to get the decomposition is not computed, it should be done next. The next step is to choose a component of the decomposition, say $\Q M$ and a basis of it. After that, $M = T^{-1}(\Q M) \cap L$ is a sublattice of $L$. The last step is to extend a basis of $M$ to $L$ (see \cite{LatticeBase} for an algorithm).
\begin{example}\label{exmp:Decompose}
Consider the group $G$ generated by 
% [4,149,1]
$$
\sigma = \left[\begin{array}{rrrrr}
0 & 0 & -1 & 0 & 1 \\
-1 & 0 & 0 & 0 & 1 \\
0 & 1 & 0 & 0 & 0 \\
0 & 0 & 1 & -1 & -1 \\
0 & 0 & -1 & 0 & 0
\end{array}\right]
 \tand
\tau = \left[\begin{array}{rrrrr}
-1 & 0 & 0 & -1 & 0 \\
0 & 0 & -1 & 0 & 1 \\
0 & 0 & 0 & 0 & -1 \\
1 & -1 & 0 & 0 & -1 \\
-1 & 0 & 0 & 0 & 0
\end{array}\right].
$$
%The $\Q$-class of $G$ is accessible in the list of rank 4 groups in \cite{Hoshi} by the name cryst4[149].
The generators of the $\Q$-class are
$$
\sigma' =\left[\begin{array}{rrr|rr}
0 & 0 & 1 & 0 & 0 \\
1 & -1 & 1 & 0 & 0 \\
0 & -1 & 0 & 0 & 0 \\
\hline
0 & 0 & 0 & 0 & 1 \\
0 & 0 & 0 & 1 & 0
\end{array}\right]
 \tand
\tau' = \left[\begin{array}{rrr|rr}
1 & -1 & 1 & 0 & 0 \\
1 & 0 & 0 & 0 & 0 \\
0 & 0 & -1 & 0 & 0 \\
\hline
0 & 0 & 0 & 0 & 1 \\
0 & 0 & 0 & -1 & -1
\end{array}\right].
$$
Considering the orders of matrices we can figure out there exist an invertible integral matrix $T$ such that 
$$T\sigma T^{-1}= \sigma'$$
$$T\tau T^{-1} = \tau'.$$
Assume 25 indeterminates $t_{00}  ,\ldots , t_{44}$ and the matrix $T=\left[t_{ij}\right]_{0 \leq i,j \leq 4}$
Then 
$$
T\sigma= \sigma' T
, \,\,\,
T\tau= \tau' T
$$
The transformation $T$ can be found by solving (and replacing parameters) the linear system obtained from above equations as
$$
T =\left[\begin{array}{rrrrr}
-1 & 1 & -2 & 1 & 1 \\
-1 & 2 & -1 & 0 & -1 \\
0 & -1 & -1 & -1 & 0 \\
1 & -2 & 1 & 0 & -1 \\
1 & 0 & -1 & 0 & 1
\end{array}\right].
$$
Let $\Q M = \langle e_1, e_2 , e_3 \rangle_\Q$.
Hence $$T^{-1}(\Q M) = \langle   \left[-\frac{1}{4},\,\frac{1}{4},\,-\frac{1}{4},\,\frac{1}{4},\,\frac{3}{4}\right], \left[-\frac{1}{4},\,\frac{1}{4},\,-\frac{1}{4},\,-\frac{1}{4},\,\frac{1}{4}\right],  \left[-\frac{1}{4},\,-\frac{1}{4},\,-\frac{1}{4},\,-\frac{1}{4},\,-\frac{1}{4}\right]\rangle_\Q .$$
Now
$$M = L \cap T^{-1}(\Q M) = \langle \left[\begin{array}{ccccc}1&0&1&0&-1\end{array}\right], \left[\begin{array}{rrrrr}
0 & 1 & 0 & 0 & 1
\end{array}\right], \left[\begin{array}{rrrrr}
0 & 0 & 0 & 1 & 1
\end{array}\right] \rangle_\Z$$ 
By extending the above basis of $M$ to a basis of $L$ (by adding $[0,0,1,0,0]$ and $[0,0,0,0,1]$) and forming the change of basis matrix we get 
$$
S =  \left[\begin{array}{rrrrr}
1 & 0 & 1 & 0 & -1 \\
0 & 1 & 0 & 0 & 1 \\
0 & 0 & 0 & 1 & 1 \\
0 & 0 & 1 & 0 & 0 \\
0 & 0 & 0 & 0 & 1
\end{array}\right]
$$
 which gives 
$$
 S
\left[\begin{array}{rrrrr}
0 & 0 & -1 & 0 & 1 \\
-1 & 0 & 0 & 0 & 1 \\
0 & 1 & 0 & 0 & 0 \\
0 & 0 & 1 & -1 & -1 \\
0 & 0 & -1 & 0 & 0
\end{array}\right]
S^{-1}
= 
\left[\begin{array}{rrr|rr}
0 & 1 & 0 & 0 & 0 \\
-1 & 0 & 0 & 0 & 0 \\
0 & 0 & -1 & 0 & 0 \\
\hline
0 & 1 & 0 & 0 & -1 \\
0 & 0 & 0 & -1 & 0
\end{array}\right]
$$
$$
 S
\left[\begin{array}{rrrrr}
-1 & 0 & 0 & -1 & 0 \\
0 & 0 & -1 & 0 & 1 \\
0 & 0 & 0 & 0 & -1 \\
1 & -1 & 0 & 0 & -1 \\
-1 & 0 & 0 & 0 & 0
\end{array}\right]
S^{-1}
= 
\left[\begin{array}{rrr|rr}
0 & 0 & -1 & 0 & 0 \\
-1 & 0 & 0 & 0 & 0 \\
0 & -1 & 0 & 0 & 0 \\
\hline
0 & 0 & 0 & 0 & -1 \\
-1 & 0 & 0 & 1 & -1
\end{array}\right]
$$
Now we can get the following exact sequence of lattices
$$0 \longrightarrow M \longrightarrow L \longrightarrow L/M \longrightarrow 0.$$ 
\end{example}

\section{Results}\label{sec:results}
The 18 maximal indecomposable stably rational lattices found in \cite{Hoshi} are divided into 4 families. The first family are the ones interpreted as lattices of root systems which are hereditarily rational. The second family contains all lattices on which Algorithm \ref{alg:Fixed Point} will not fail. The third family contains lattices on which  Algorithm \ref{alg:Sign Fixed Point} does not fail while Algorithm \ref{alg:Fixed Point} fails. The last family contains all lattices on which either, both Algorithms \ref{alg:Fixed Point} and \ref{alg:Sign Fixed Point} fail but still the general idea of reduction works, or they are irreducible.
All lattices of the first family are hereditarily rational. The second family contains lattices of which after reduction, rationality of the reduced component is unknown. By arguments in the previous sections we have proved the following theorem.
\begin{theorem}
The groups presented in Table \ref{Tbl:Rational} are hereditarily rational.
\end{theorem}
 \begin{table}  
\centering
\begin{tabular}{lllll}
CARAT ID & Group Structure & $\#G$ & Description.\\\hline
$(5,942,1)$ & ${\rm Imf}(5,1,1)$ & $3840$ & The root lattice of $B_5$ \\
 $(5,953,4)$ & $\mathrm{S}_6$ & $720$ &  The root lattice of $A_5$ \\
 $(5,726,4)$ & $C_2^4\rtimes \mathrm{S}_4$ & $384$   & reduced component $[4,32,21,1]$  \\
$(5,911,4)$ & $\mathrm{S}_5$ & $120$  & reduced component $[ 4, 31, 4, 1 ]$  \\
 $(5,341,6)$ & $D_8\times \mathrm{S}_3$ & $48$  & reduced component $[ 4, 20, 17, 2 ]$ &\\
 $(5,531,13)$ & $C_2\times \mathrm{S}_4$ & $48$  & reduced component $[ 4, 25, 9, 2 ]$ &\\
 $(5,245,12)$ & $C_2^2\times \mathrm{S}_3$ & $24$ &reduced component $[ 4, 14, 10, 2 ]$&
\end{tabular}
\caption{Hereditarily rational groups among the maximal $18$ groups 
found in \cite{Hoshi}.}
\label{Tbl:Rational}
\end{table}
\noindent
The exceptional cases of the second family are presented in Table \ref{Tbl:RationalityUnknown}. In each case the reduced component is stably rational as proved in \cite{Hoshi}. Their rationality is unknown yet. In \cite{Nicole1} the author has proved that subgroups of $[ 4, 25, 8, 5 ]$ are rational except for possibly $$[4, 6, 2, 11], [4, 12, 4, 13], [4, 13, 2, 6], [4, 13, 3, 6], [4, 13, 7, 12], [4, 24, 4, 6], [4, 25, 4, 5], [4, 25, 8, 5].$$
There will be precisely one subgroup of $G_{14}$ with each of the dimension 4 reduced components in the above list, so except for possibly those subgroups of $G_{14}$ the rest are rational. The exceptional cases are presented in Table \ref{ExceptionalG14}.

From the above list $$[4, 6, 2, 11], [4, 12, 4, 13], [4, 13, 2, 6], [4, 13, 3, 6], [4, 13, 7, 12]$$ are subgroups of $[4, 13, 7, 12]$ which means we have the same problem for cases $G_{17}$ and $G_{18}$. Hence except for possibly the subgroups of $G_{17}$ and $G_{18}$ associated to above list their rest of subgroups are rational. For the exceptional cases see Table \ref{ExceptionalG17} and Table \ref{ExceptionalG18}.
 \begin{table}
\centering
\begin{tabular}{lllll}
CARAT ID & Group Structure & $\#G$ & Description.\\\hline
 $(5,533,8)$ & $C_2\times \mathrm{S}_4$ & $48$  & reduced component $[ 4, 25, 8, 5 ]$ &\\
 $(5,81,42)$ & $C_2\times D_8$ & $16$  & reduced component $[ 4, 13, 7, 12 ]$&\\
 $(5,81,48)$ & $C_2\times D_8$ & $16$  & reduced component $[ 4, 13, 7, 12 ]$ &
\end{tabular}
\caption{Groups among 18 maximals which are reduced but rationality of rank 4 sublattice is unknown}
\label{Tbl:RationalityUnknown} 
\end{table}
\begin{table}
\centering
\begin{tabular}{llll}
CARAT ID & Group Structure & $\#G$ & Description.\\\hline
$( 5, 32, 52)$ &	$C_2 \times C_2 \times C_2$& 8 &reduced component 	[4, 6, 2, 11]\\
$( 5, 99, 53 )$ &	$D_8$ & 8&	reduced component [4, 12, 4, 13]\\
$( 5, 103, 22 )$ & $C_4 \times C_2$& 8 &reduced component [4, 13, 2, 6]\\
$( 5, 98, 22 )$& 	$D_8$ & 8&	reduced component [4, 13, 3, 6]\\
$( 5, 81, 50 )$ &		$C_2 \times D_8$ & 16& reduced component [4, 13, 7, 12]\\
$( 5, 522, 15)$ &	$\mathrm{S}_4$& 24&	reduced component 	[4, 24, 4, 6]\\
$( 5, 521, 15 )$ &		$\mathrm{S}_4$ &24&		reduced component [4, 25, 4, 5]\\
$( 5, 533, 8 )$&		$C_2 \times	 \mathrm{S}_4$ & 48&		reduced component [4, 25, 8, 5]
\end{tabular}
\caption{Subgroups of $G_{14}$ that have associated tori which are stably rational but whose rationality is unknown.}
\label{ExceptionalG14}
\end{table}
\begin{table}
\centering
\begin{tabular}{llll}
CARAT ID & Group Structure & $\#G$ & Description.\\\hline
$(5, 32, 49)$ &	$C_2 \times C_2 \times C_2$& 8 &reduced component [4, 6, 2, 11]\\
$( 5, 99, 52)$ &	$D_8$ & 8&	reduced component 	 [4, 12, 4, 13]\\
$(  5, 103, 16)$ & $C_4 \times C_2$& 8 &		reduced component  [4, 13, 2, 6]\\
$(  5, 98, 16 )$& 	$D_8$ & 8&		reduced component  [4, 13, 3, 6]\\
$(  5, 81, 42 )$ &		$C_2 \times D_8$ & 16& reduced component  [4, 13, 7, 12]
\end{tabular}
\caption{Subgroups of $G_{17}$ that have associated tori which are stably rational but whose rationality is unknown.}
\label{ExceptionalG17}
\end{table}
\begin{table}
\centering
\begin{tabular}{llll}
CARAT ID & Group Structure & $\#G$ & Description.\\\hline
$(5, 32, 46)$ &	$C_2 \times C_2 \times C_2$& 8 &reduced component [4, 6, 2, 11]\\
$( 5, 99, 54 )$ &	$D_8$ & 8&		reduced component  [4, 12, 4, 13]\\
$( 5, 103, 24 )$ & $C_4 \times C_2$& 8 &	reduced component 	 [4, 13, 2, 6]\\
$( 5, 98, 24)$& 	$D_8$ & 8&		reduced component  [4, 13, 3, 6]\\
$( 5, 81, 48 )$ &		$C_2 \times D_8$ & 16&reduced component  [4, 13, 7, 12]
\end{tabular}
\caption{Subgroups of $G_{18}$ that have associated tori which are stably rational but whose rationality is unknown.}
\label{ExceptionalG18}
\end{table}
\begin{theorem}
All subgroups of $G_{14},G_{17}$ and $G_{18}$ are rational except for possibly the subgroups in Table \ref{ExceptionalG14}, Table \ref{ExceptionalG17}  and Table \ref{ExceptionalG18}.
\end{theorem}
There are 4 cases, namely $G_4, G_5,G_6$ and $G_{11}$, in which after the reduction we get a rank one sign permutation lattice (more information is given in Table \ref{SignPerm}). The same also happened in some subgroups of irreducible lattices (see Table \ref{SubIrreducible}). It is possible that these groups are hereditarily rational, but we do not currently have a proof. One possible approach to prove rationality in these cases may be the following argument.

The following table summarizes the information about reduction of lattices in the third family.
 \begin{table}[H]
\centering
\begin{tabular}{lllll} 
 CARAT ID & $G$ & $\#G$ & Description.\\\hline
 $(5,919,4)$ & $C_2\times \mathrm{S}_5$ & $240$  &  reduced component $[4,31,7,1]$ \\
 $(5,801,3)$ & $C_2\times (\mathrm{S}_3^2\rtimes C_2)$ & $144$  &reduced component $[ 4, 29, 9, 2 ]$  \\
 $(5,655,4)$ & $D_8^2\rtimes C_2$ & $128$  & reduced component $[4,32,17,1]$    \\
 $(5,337,12)$ & $D_8\times \mathrm{S}_3$ & $48$  &  reduced component $[ 4, 20, 20, 4 ]$   &\\
\end{tabular}
\caption{The groups corresponding to maximal stably rational tori of dimension 5 whose associated lattices are indecomposable and have a rank 1 sign quotient.}
\label{SignPerm}
\end{table}
For the last family we have considered their maximal subgroups and we could not decide about the rationality of the groups presented in the following table.
\begin{theorem}\label{SubG8}
All groups in Table \ref{IrreducibleSubs} are hereditarily rational. That is, all subgroups of $G_8$, $G_9$, $G_{10}$ and $G_{15}$ except for possibly the subgroups in Table \ref{SubIrreducible} are hereditarily rational.
\end{theorem}
A proof of Theorem \ref{SubG8} is provided in Appendix A.
\begin{center} 
\begin{table} 
\begin{tabular}{cccc}
 Carat ID  & $G$ & $\#G$ & Description\\
\hline
 $[5,173,4]$ & $\mathrm{S}_3$ & 6 &  reduced comp. $[ 4, 17, 1, 1 ]$,  rank 1 sign perm. quot. \\
 $[5,391,4]$ &$D_{12}$ & $12$ &  reduced comp. $ [ 4, 21, 3, 1 ]$  rank 1 sign perm. quot. \\
 $[5,461,4]$ &$C^2_2 \times \mathrm{S}_3$ & 24 & reduced comp.$[3,6,7,1]$, quot $[2,4,4,1]$  \\
 $[5,580,4]$ & $A_4$ & $12$ &  reduced comp.$[3,7,1,1]$, quot $[ 2, 4, 1, 1 ]$\\
 $[5,606,4]$ & $C_2 \times A_4$ & 24 &reduced comp. $[3,7,2,1]$, quot $[ 2, 4, 1, 1 ]$ \\
 $[5,607,4]$ & $\mathrm{S}_4$ & 24 &reduced comp. $[3,7,4,1]$, quot $[ 2, 4, 2, 1 ]$ \\
 $[5,607,9]$ &$\mathrm{S}_4$ & 24 & reduced comp. $ [ 3, 7, 4, 1 ]$, quot $[ 2, 4, 2, 2 ]$ \\
 $[5,608,4]$ &$\mathrm{S}_4$ & 24 & reduced comp.t $[ 3, 7, 3, 1 ]$ , quot $[ 2, 4, 2, 1 ]$ \\
 $[5,917,3]$ &$C_5 \rtimes C_4$ & 20 &  reduced comp.  $[ 4, 31, 1, 1 ]$  rank 1 sign perm. quot. \\
 $[5,917,4]$ &$C_5 \rtimes C_4$ & 20 &  reduced comp.  $[ 4, 31, 1, 1 ]$   rank 1 sign perm. quot. \\
 $[5,623,4]$ &$C_2 \times \mathrm{S}_4$ & 48 & reduced comp. $[3,7,5,1]$, quot $[ 2, 4, 2, 1 ]$ \\
 $[5,952,2]$ &$A_5$ & 60 & absolutely irreducible \\
 $[5,952,4]$ &$A_5$ & 60 & absolutely irreducible \\
 $[5,946,2]$ &$\mathrm{S}_5$ & 120 & absolutely irreducible \\
 $[5,946,4]$ &$\mathrm{S}_5$ & 120 & absolutely irreducible \\
 $[5,947,2]$ &$\mathrm{S}_5$ & 120 & absolutely irreducible 
\end{tabular}
\caption{Subgroups of $G_8, G_9, G_{10}$ and $G_{15}$ that have associated tori which are stably rational but whose rationality is unknown.}
\label{SubIrreducible}
\end{table}
\end{center}
The following table present the reduced components of the subgroups mentioned in Theorem \ref{SubG8}.
\begin{table}
\begin{tabular}{llllll} 
Number & CARAT ID & $G$ & $\#G$ & Description.\\\hline
1 & $[5,6,3]$ & $C_2$ & 2  & [ 4, 2, 2, 2 ] \\
2 & $[5,18,28]$ & $C_2 \times C_2$  &  4& [ 4, 4, 3, 4 ] \\
3 & $[5,19,14]$ &  $C_2 \times C_2$  &  4  & [ 4, 5, 1, 10 ]  \\
4 & $[5,22,14]$ & $C_2 \times C_2 \times C_2$  & 8 & [ 4, 6, 1, 9 ] \\
5 & $[5,57,8]$ & $C_4$ &  4 & [ 4, 7, 1, 2 ] \\
6 & $[5,81,54]$ & $C_2 \times D_8$  & 16 & [ 4, 13, 7, 5 ]  \\
7 & $[5,98,28]$ & $D_8$ & 8 & [ 4, 13, 3, 3 ] \\
8 & $[5,99,57]$ & $D_8$ & 8  & [ 4, 12, 4, 7 ]   \\
9 & $[5,164,2]$ & $C_3$ & 3 & [ 4, 11, 1, 1 ]  \\
10 & $[5,174,2]$ & $\mathrm{S}_3$ & 6  & [ 4, 17, 1, 3 ] \\
11 & $[5,174,5]$ & $\mathrm{S}_3$ & 6 & [ 4, 17, 1, 2 ] \\
12 & $[5,389,4]$ & $D_{12}$ &  12 &  [ 4, 21, 3, 2 ] \\
13 & $[5,901,3]$ & $D_{10}$  & 10 & [ 4, 27, 3, 1 ] \\
14 & $[5,918,4]$ & $C_5 \rtimes C_4$ & 20 & [4, 31, 1, 2]
\end{tabular}
\caption{Hereditarily rational subgroups of $G_8, G_9, G_{10}$ and $G_{15}$.}
\label{IrreducibleSubs}
\end{table}
\begin{remark}
The union of the set of groups in Table \ref{SubIrreducible} and the set of subgroups of the groups in Table \ref{IrreducibleSubs}, is the set of all subgroups of $G_8, G_9, G_{10}$ and $G_{15}$.
\end{remark}
Appendix B presents tables of conjugacy classes of indecomposable subgroups of $\mathrm{GL}(5,\Z)$ which correspond to stably rational tori of dimension 5 from Hoshi and Yamasaki's list. From this list, those stably rational tori of dimension 5 whose rationality is unknown are listed.

\bibliographystyle{plain} % (change according to your preference)
%%%% ***   Set the bibliography file.   ***
\bibliography{bibliography}{}
%%%
\end{document}
