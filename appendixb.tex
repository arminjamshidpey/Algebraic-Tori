\chapter{A Proof of Theorem \ref{SubG8}}\label{AppB}
This chapter is devoted to a concrete proof of Theorem \ref{IrreducibleSubs}. 
We use simillar methods to the ones we already applied in chapter 3. After 
reducing the lattices we compare the sublattices with the rational ones introduced 
in \cite{Kunyavski} and \cite{Nicole1}.
 \section{ (5,6,3 ) }
The group is generated by  
$$
\left[ \begin {array}{ccccc} -1&0&0&0&0\\0&0&0&0&-
1\\0&0&0&-1&0\\0&0&-1&0&0
\\0&-1&0&0&0\end {array} \right]. 
$$
The corresponding lattice is sign permutation. This implies rationality of 
the corresponding torus.
 \section{(5,18,28)}
The group is generated by 
$$
 \left[ \begin {array}{ccccc} 1&0&0&0&0\\0&0&1&0&0
\\0&1&0&0&0\\0&0&0&0&1
\\0&0&0&1&0\end {array} \right] 
\tand
 \left[ \begin {array}{ccccc} -1&0&0&0&0\\0&0&0&0&1
\\0&0&0&1&0\\0&0&1&0&0
\\0&1&0&0&0\end {array} \right] 
$$

The corresponding lattice is a sign permutation lattice. Thus it is hereditarily rational.

 \section{(5,19,14)}
Algorithm 2 produces the change of basis matrix 
$$
\left[ \begin {array}{ccccc} 0&1&0&0&0\\ 0&0&1&0&0
\\ 0&0&0&1&0\\ 0&0&0&0&1
\\ 1&0&1&0&0\end {array} \right] 
$$
With the above transformation we can see the new representative is generated by 
$$
 \left[ \begin {array}{cccc|c} 0&0&1&-1&0\\0&-1&0&-2
&0\\1&0&0&1&0\\0&0&0&1&0
\\ \hline 0&1&0&1&1\end {array} \right] 
 \left[ \begin {array}{cccc|c} 0&0&1&0&0\\0&1&0&2&0
\\1&0&0&0&0\\0&0&0&-1&0
\\\hline 0&0&0&-1&1\end {array} \right] 
$$
we can produce
$$
\exactseq{}
.$$
The corresponding group to $M$ has GAP ID [4,5,1,10] and also can be generated by
$$
 \left[ \begin {array}{ccc|c} 1&0&-1&0\\ 0&1&-1&0
\\ 0&0&-1&0\\ \hline 0&0&0&-1\end {array}
 \right] 
 \tand
 \left[ \begin {array}{ccc|c} 0&1&0&0\\ 1&0&0&0
\\ 0&0&1&0\\  \hline 0&0&0&-1\end {array}
 \right] 
$$
So $M$ decomposes into a direct sum of a rank one sign permutation lattice 
(which is hereditarily rational) and a rank 3 lattice given by a group, $H$, 
generated by
$$
\left[ \begin {array}{cc|c} 1&0&-1\\ 0&1&-1
\\ \hline 0&0&-1\end {array}
 \right] 
 \tand
 \left[ \begin {array}{cc|c} 0&1&0\\ 1&0&0
\\ \hline 0&0&1\end {array}
 \right] 
$$
Looking at the generators of $H$ tells us we can form 
$$0 \longrightarrow \Z^- \longrightarrow L_H \longrightarrow P \longrightarrow 0$$ 
where $P$ is given by the group generated by
$$
\begin{bmatrix}
1&0\\
0& 1
\end{bmatrix}
\tand
\begin{bmatrix}
0&1\\
1&0
\end{bmatrix}
.$$
Since $P$ is a permutation lattice, by Corollary \ref{permcoker} we conclude 
that [4,5,1,10] is hereditarily rational. This implies our desired result 
which is hereditarily rationality of (5,19,14). 


 \section{(5,22,14)}
The group is generated by 
$$
  \left[ \begin {array}{cccc|c} 1&0&0&0&-1\\0&1&0&0&1
\\0&0&1&0&1\\0&0&0&1&-1
\\ \hline 0&0&0&0&-1\end {array} \right] ,
 \left[ \begin {array}{cccc|c} 1&0&0&0&0\\0&0&0&1&-1
\\0&0&1&0&0\\0&1&0&0&1
\\ \hline 0&0&0&0&1\end {array} \right] 
\tand
 \left[ \begin {array}{cccc|c} 0&0&1&0&1\\0&1&0&0&0
\\1&0&0&0&-1\\0&0&0&1&0
\\ \hline 0&0&0&0&1\end {array} \right] 
$$
Now we define $P$ to be the lattice corresponding to, $H$, generated by 
$$
  \left[ \begin {array}{cccc} 1&0&0&0\\0&1&0&0
\\0&0&1&0\\0&0&0&1
\end {array} \right] ,
 \left[ \begin {array}{cccc} 1&0&0&0\\0&0&0&1
\\0&0&1&0\\0&1&0&0
\end {array} \right] 
\tand
 \left[ \begin {array}{cccc} 0&0&1&0\\0&1&0&0
\\1&0&0&0\\0&0&0&1
\end {array} \right],
$$
We can see the corresponding lattice to (5,22,14), $L$, fits into the following exact sequence
$$0\longrightarrow \Z^- \longrightarrow L \longrightarrow P \longrightarrow 0.$$
and since $P$ is permutation, by Corollary \ref{permcoker} we can conclude that $L$ is hereditarily rational.
 \section{(5,57,8)}
The group is generated by 
$$
\left[ \begin {array}{ccccc} -1&0&0&0&0\\0&0&0&1&0
\\0&-1&0&0&0\\0&0&0&0&1
\\0&0&-1&0&0\end {array} \right] 
$$
and the corresponding lattice is a sign permutation lattice which is hereditarily rational.
  \section{(5,81,54)}
The group is generated by 
$$
 \left[ \begin {array}{ccccc} 1&0&1&0&1\\0&1&0&0&0
\\0&0&0&0&-1\\0&0&1&1&1
\\0&0&-1&0&0\end {array} \right],
 \left[ \begin {array}{ccccc} 0&0&1&0&0\\-1&0&0&1&0
\\1&0&0&0&0\\0&1&1&0&0
\\-1&0&-1&0&-1\end {array} \right]
\tand
 \left[ \begin {array}{ccccc} 0&0&-1&-1&-1\\0&1&0&0
&0\\0&0&1&0&0\\-1&0&-1&0&-1
\\0&0&0&0&1\end {array} \right] 
$$
The dual group is generated by 
$$
  \left[ \begin {array}{ccccc} 1&0&0&0&0\\0&1&0&0&0
\\1&0&0&1&-1\\0&0&0&1&0
\\1&0&-1&1&0\end {array} \right] ,
 \left[ \begin {array}{ccccc} 0&-1&1&0&-1\\0&0&0&1&0
\\1&0&0&1&-1\\0&1&0&0&0
\\0&0&0&0&-1\end {array} \right] 
\tand
 \left[ \begin {array}{ccccc} 0&0&0&-1&0\\0&1&0&0&0
\\-1&0&1&-1&0\\-1&0&0&0&0
\\-1&0&0&-1&1\end {array} \right] 
$$
Algorithm (2) produces the change of basis matrix 
$$
 \left[ \begin {array}{ccccc} 0&1&0&0&0\\ 0&0&1&0&0
\\ 0&0&0&1&0\\ 0&0&0&0&1
\\ 1&-2&1&-1&-1\end {array} \right] 
$$
With the above transformation we can see the new representative is generated by 
$$
 \left[ \begin {array}{cccc|c} 1&2&0&2&0\\0&-1&0&-2&0
\\0&2&1&2&0\\0&0&0&1&0
\\ \hline 0&1&0&1&1\end {array} \right], 
 \left[ \begin {array}{cccc|c} 0&2&1&0&0\\0&-1&0&0&0
\\1&2&0&0&0\\0&0&0&-1&0
\\ \hline 0&1&0&0&1\end {array} \right] 
\tand
 \left[ \begin {array}{cccc|c} 1&-2&-2&-2&0\\0&2&1&1
&0\\0&-2&-1&-2&0\\0&-1&-1&0&0
\\ \hline 0&-1&-1&-1&1\end {array} \right] 
$$
Now by considering $M$ to be the corresponding lattice to 
$$
 \left[ \begin {array}{cccc} 1&2&0&2\\0&-1&0&-2
\\0&2&1&2\\0&0&0&1\end {array}
 \right], 
 \left[ \begin {array}{cccc} 0&2&1&0\\0&-1&0&0
\\1&2&0&0\\0&0&0&-1\end {array}
 \right] 
 \tand
 \left[ \begin {array}{cccc} 1&-2&-2&-2\\0&2&1&1
\\0&-2&-1&-2\\0&-1&-1&0
\end {array} \right] 
$$
we can produce
$$
\exactseq{}
.$$
The generators of [4,13,7,5] (another representative of the corresponding conjugacy class to $M$) are 
$$
\left[ \begin {array}{ccc|c} 0&-1&1&0\\ 0&-1&0&0
\\ 1&-1&0&0\\ \hline 0&0&0&-1\end {array}
 \right], 
 \left[ \begin {array}{ccc|c} 0&1&-1&0\\ 1&0&-1&0
\\ 0&0&-1&0\\ \hline 0&0&0&1\end {array}
 \right] 
 \tand
 \left[ \begin {array}{ccc|c} 0&0&-1&0\\ 1&0&-1&0
\\ 0&1&-1&0\\ \hline 0&0&0&1\end {array}
 \right] 
$$
The generators of rank 3 lattice are 
$$
 \left[ \begin {array}{ccc} 0&-1&1\\ 0&-1&0
\\ 1&-1&0\end {array} \right], 
 \left[ \begin {array}{ccc} 0&1&-1\\ 1&0&-1
\\ 0&0&-1\end {array} \right]
\tand 
 \left[ \begin {array}{ccc} 0&0&-1\\ 1&0&-1
\\ 0&1&-1\end {array} \right] 
$$
the CrystCatZClass of the former group is [3,4,6,4] which is rational by \cite{Kunyavski} and its subgroups are  [ 3, 1, 1, 1 ], [ 3, 2, 1, 2 ], [ 3, 2, 2, 2 ], [ 3, 3, 1, 4 ], [ 3, 3, 2, 4 ], [ 3, 4, 2, 2 ] and  [ 3, 4, 6, 4 ] where all of them are rational. This implies that (5, 81, 54) is hereditarily rational.
  \section{(5,98,28)}
The group is generated by 
$$
 \left[ \begin {array}{ccccc} 0&-1&0&0&0\\-1&0&0&0&0
\\-1&0&1&0&-1\\0&-1&0&1&1
\\-1&1&0&0&-1\end {array} \right] 
\tand
 \left[ \begin {array}{ccccc} 1&0&0&0&0\\0&1&0&0&0
\\0&1&-1&0&0\\0&0&1&0&-1
\\0&1&-1&-1&0\end {array} \right] 
$$
The dual group is generated by 
$$
 \left[ \begin {array}{ccccc} 0&-1&-1&0&-1\\-1&0&0&
-1&1\\0&0&1&0&0\\0&0&0&1&0
\\0&0&-1&1&-1\end {array} \right] 
\tand
 \left[ \begin {array}{ccccc} 1&0&0&0&0\\0&1&1&0&1
\\0&0&-1&1&-1\\0&0&0&0&-1
\\0&0&0&-1&0\end {array} \right] 
$$
Algorithm (1) produces the change of basis matrix 
$$
 \left[ \begin {array}{ccccc} 1&0&-1&0&0\\ 0&1&-1&0&0
\\ 0&0&0&1&0\\ 0&0&0&0&1
\\ 2&-2&-1&1&-2\end {array} \right] 
$$
With the above transformation we can see the new representative is generated by 
$$
  \left[ \begin {array}{cccc|c} -6&-5&0&-2&0\\5&4&0&2
&0\\-3&-3&1&0&0\\5&5&0&1&0
\\ \hline 3&2&0&1&1\end {array} \right] 
\tand
 \left[ \begin {array}{cccc|c} 5&6&0&0&0\\-4&-5&0&0&0
\\1&2&0&-1&0\\-3&-4&-1&0&0
\\ \hline -2&-3&0&0&1\end {array} \right] 
$$
Now by considering $M$ to be the corresponding lattice to 
$$
\left[ \begin {array}{cccc} -6&-5&0&-2\\5&4&0&2
\\-3&-3&1&0\\5&5&0&1\end {array}
 \right] 
 \tand
 \left[ \begin {array}{cccc} 5&6&0&0\\-4&-5&0&0
\\1&2&0&-1\\-3&-4&-1&0
\end {array} \right] 
$$
we can produce
$$
\exactseq{}
.$$
The generators of [4,13,3,3] are 
$$
 \left[ \begin {array}{ccc|c} 0&0&1&0\\ 0&1&0&0
\\ 1&0&0&0\\ \hline 0&0&0&-1\end {array}
 \right] 
 \left[ \begin {array}{ccc|c} 0&0&-1&0\\ 1&0&-1&0
\\ 0&1&-1&0\\ \hline 0&0&0&-1\end {array}
 \right] 
$$
The generators of rank 3 lattice are 
$$
  \left[ \begin {array}{ccc} 0&0&1\\ 0&1&0
\\ 1&0&0\end {array} \right] 
\tand
 \left[ \begin {array}{ccc} 0&0&-1\\ 1&0&-1
\\ 0&1&-1\end {array} \right] 
$$
the GAP ID of the former group is [3,4,6,4] which is hereditarily rational by the argument given in the previous case. So (5,98,28) is hereditarily rational.
  \section{(5,99,57)}
The group is generated by 
$$
 \left[ \begin {array}{ccccc} 1&0&1&0&1\\0&1&0&0&0
\\0&-1&-1&0&0\\0&0&1&1&1
\\0&1&0&0&-1\end {array} \right] 
\tand
 \left[ \begin {array}{ccccc} 0&1&1&0&0\\1&0&0&-1&0
\\0&0&0&1&0\\0&0&1&0&0
\\0&0&-1&-1&-1\end {array} \right] 
$$
The dual group is generated by  
$$
\left[ \begin {array}{ccccc} 1&0&0&0&0\\0&1&-1&0&1
\\1&0&-1&1&0\\0&0&0&1&0
\\1&0&0&1&-1\end {array} \right] 
\tand
 \left[ \begin {array}{ccccc} 0&1&0&0&0\\1&0&0&0&0
\\1&0&0&1&-1\\0&-1&1&0&-1
\\0&0&0&0&-1\end {array} \right] 
$$
Algorithm (1) produces the change of basis matrix 
$$
\left[ \begin {array}{ccccc} 0&1&0&0&0\\ 0&0&1&0&0
\\ 0&0&0&1&0\\ 0&0&0&0&1
\\ 1&2&-1&-1&1\end {array} \right] 
$$
With the above transformation we can see the new representative is generated by 
$$
\left[ \begin {array}{cccc|c} 1&-2&0&-2&0\\-1&0&0&1
&0\\0&2&1&2&0\\1&-1&0&-2&0
\\ \hline 0&1&0&1&1\end {array} \right] 
\tand
 \left[ \begin {array}{cccc|c} -2&-2&-1&0&0\\1&1&1&0
&0\\1&2&0&0&0\\-1&-2&-1&-1&0
\\ \hline 1&1&0&0&1\end {array} \right] 
$$
Now by considering $M$ to be the corresponding lattice to 
$$
\left[ \begin {array}{cccc} 1&-2&0&-2\\-1&0&0&1
\\0&2&1&2\\1&-1&0&-2\end {array}
 \right] 
 \tand
 \left[ \begin {array}{cccc} -2&-2&-1&0\\1&1&1&0
\\1&2&0&0\\-1&-2&-1&-1
\end {array} \right] 
$$
we can produce
$$
\exactseq{}
.$$
The generators of [4,12,4,7] are 
$$
 \left[ \begin {array}{ccc|c} 0&0&1&0\\ 0&1&0&0
\\ 1&0&0&0\\ \hline 0&0&0&-1\end {array}
 \right] 
 \left[ \begin {array}{ccc|c} 0&0&-1&0\\ 1&0&-1&0
\\ 0&1&-1&0\\ \hline 0&0&0&1\end {array}
 \right] 
$$
The generators of rank 3 lattice are 
$$
 \left[ \begin {array}{ccc} 0&0&1\\ 0&1&0
\\ 1&0&0\end {array} \right] 
 \left[ \begin {array}{ccc} 0&0&-1\\ 1&0&-1
\\ 0&1&-1\end {array} \right]
$$
the GAP ID of the former group is [3,4,6,4] which is hereditarily rational by argument given in the previous case. So (5,99,57) is hereditarily rational.
  \section{(5,164,2)}
The group is generated by 
$$
 \left[ \begin {array}{cc|ccc} -1&1&0&0&0\\-1&0&0&0&0
\\ \hline 0&0&0&-1&0\\0&0&0&0&1
\\0&0&-1&0&0\end {array} \right] 
$$
The corresponding lattice decomposes into a rank 2 lattice which is hereditarily rational and a rank 3 sign permutation lattice which is also hereditarily rational. Hence (5,164,2) is hereditarily rational.
  \section{(5,174,2)}
The group is generated by 
$$
  \left[ \begin {array}{ccccc} 0&0&-1&0&-1\\0&0&0&1&
1\\-1&0&0&0&-1\\0&1&0&0&-1
\\0&0&0&0&1\end {array} \right] 
\tand
 \left[ \begin {array}{ccccc} 0&0&-1&0&-1\\-1&0&1&0
&0\\0&0&-1&1&0\\0&0&-1&0&0
\\0&1&1&0&0\end {array} \right] 
$$
The dual group is generated by 
$$
 \left[ \begin {array}{ccccc} 0&0&-1&0&0\\0&0&0&1&0
\\-1&0&0&0&0\\0&1&0&0&0
\\-1&1&-1&-1&1\end {array} \right] 
\tand
 \left[ \begin {array}{ccccc} 0&-1&0&0&0\\0&0&0&0&1
\\-1&1&-1&-1&1\\0&0&1&0&0
\\-1&0&0&0&0\end {array} \right] 
$$
Algorithm (1) produces the change of basis matrix 
$$
 \left[ \begin {array}{ccccc} 1&0&0&0&0\\ 0&0&1&0&0
\\ 0&0&0&1&0\\ 0&0&0&0&1
\\ 1&-1&0&0&-1\end {array} \right] 
$$
With the above transformation we can see the new representative is generated by 
$$
\left[ \begin {array}{cccc|c} 0&-1&1&0&0\\-1&0&0&-1
&0\\0&0&0&-1&0\\0&0&-1&0&0
\\ \hline 0&0&-1&-1&1\end {array} \right] 
\tand
 \left[ \begin {array}{cccc|c} -1&0&0&-1&0\\0&-1&1&0
&0\\0&-1&0&0&0\\1&0&0&0&0
\\ \hline 1&-1&0&0&1\end {array} \right] 
$$
Now by considering $M$ to be the corresponding lattice to 
$$
 \left[ \begin {array}{cccc} 0&-1&1&0\\-1&0&0&-1
\\0&0&0&-1\\0&0&-1&0\end {array}
 \right] 
 \tand
 \left[ \begin {array}{cccc} -1&0&0&-1\\0&-1&1&0
\\0&-1&0&0\\1&0&0&0\end {array}
 \right] 
$$
we can produce
$$
\exactseq{}
.$$
The generators of [4,17,13] are 
$$
 \left[ \begin {array}{cc|cc} -1&-1&0&0\\ 0&1&0&0
\\ \hline 0&0&1&1\\  0&0&0&-1\end {array}
 \right] 
 \tand
 \left[ \begin {array}{cc|cc} -1&-1&0&0\\ 1&0&0&0
\\ \hline 0&0&-1&-1\\ 0&0&1&0\end {array}
 \right] 
$$
and the lattice decomposes into rank 2 lattices which we know they are hereditarily rational. This implies that (5,174,2) is hereditarily rational.
  \section{(5,174,5)}
The group is generated by 
$$
 \left[ \begin {array}{cc|ccc} -1&0&0&0&0\\1&1&0&0&0
\\ \hline 0&0&0&0&1\\0&0&0&1&0
\\0&0&1&0&0\end {array} \right] 
\tand
 \left[ \begin {array}{cc|ccc} -1&-1&0&0&0\\1&0&0&0&0
\\ \hline 0&0&0&0&1\\0&0&-1&0&0
\\0&0&0&-1&0\end {array} \right] 
$$
and the lattice decomposes into a rank 2 lattice and a rank 3 sign permutation lattice, both of which are hereditarily rational. This implies that (5,174,5) is hereditarily rational.
  \section{(5,389,4)}
The group is generated by 
$$
 \left[ \begin {array}{ccccc} 1&0&0&-1&0\\0&1&0&-1&0
\\0&0&0&-1&-1\\0&0&0&-1&0
\\0&0&-1&1&0\end {array} \right] 
\tand
 \left[ \begin {array}{ccccc} 1&0&0&-1&0\\1&0&0&0&1
\\1&-1&0&0&0\\1&0&-1&0&0
\\-1&0&0&0&0\end {array} \right] 
$$
The dual group is generated by 
$$
 \left[ \begin {array}{ccccc} 1&0&0&0&0\\0&1&0&0&0
\\0&0&0&0&-1\\-1&-1&-1&-1&1
\\0&0&-1&0&0\end {array} \right] 
\tand
 \left[ \begin {array}{ccccc} 1&1&1&1&-1\\0&0&-1&0&0
\\0&0&0&-1&0\\-1&0&0&0&0
\\0&1&0&0&0\end {array} \right] 
$$
Algorithm (1) produces the change of basis matrix 
$$
 \left[ \begin {array}{ccccc} 0&1&0&0&0\\ 0&0&1&0&0
\\ 0&0&0&1&0\\ 0&0&0&0&1
\\ 1&0&1&0&-1\end {array} \right] 
$$
With the above transformation we can see the new representative is generated by 
$$
 \left[ \begin {array}{cccc|c} 1&0&-1&0&0\\0&0&0&-1&0
\\0&0&-1&0&0\\0&-1&0&0&0
\\ \hline 0&0&-1&0&1\end {array} \right]
\tand
 \left[ \begin {array}{cccc|c} 0&0&0&1&0\\-1&0&1&0&0
\\0&-1&0&0&0\\0&0&-1&0&0
\\ \hline 0&0&-1&0&1\end {array} \right] 
$$
Now by considering $M$ to be the corresponding lattice to 
$$
 \left[ \begin {array}{cccc} 1&0&-1&0\\0&0&0&-1
\\0&0&-1&0\\0&-1&0&0\end {array}
 \right] 
 \tand
 \left[ \begin {array}{cccc} 0&0&0&1\\-1&0&1&0
\\0&-1&0&0\\0&0&-1&0\end {array}
 \right] 
$$
we can produce
$$
\exactseq{}
.$$
the above group is [4,21,3,2] which has the following subgroups [ 4, 1, 1, 1 ], [ 4, 3, 1, 3 ], [ 4, 5, 1, 1 ], [ 4, 11, 1, 1 ], [ 4, 17, 1, 2 ], [ 4, 17, 1, 3 ], [ 4, 21, 1, 1 ] and [ 4, 21, 3, 2 ] 
where all of them are rational.

  \section{(5,901,3)}
The group is generated by 
$$
 \left[ \begin {array}{ccccc} 0&1&0&0&0\\1&0&0&0&0
\\0&0&0&0&-1\\0&0&0&1&0
\\0&0&-1&0&0\end {array} \right] 
\tand
 \left[ \begin {array}{ccccc} 0&0&0&0&-1\\1&0&0&0&0
\\0&1&0&0&0\\0&0&1&0&0
\\0&0&0&-1&0\end {array} \right] 
$$
The dual group is generated by 
$$
 \left[ \begin {array}{ccccc} 0&1&0&0&0\\1&0&0&0&0
\\0&0&0&0&-1\\0&0&0&1&0
\\0&0&-1&0&0\end {array} \right] 
\tand
 \left[ \begin {array}{ccccc} 0&1&0&0&0\\0&0&1&0&0
\\0&0&0&1&0\\0&0&0&0&-1
\\-1&0&0&0&0\end {array} \right] 
$$
Algorithm (1) produces the change of basis matrix 
$$
 \left[ \begin {array}{ccccc} 1&0&0&0&0\\ 0&0&1&0&0
\\ 0&0&0&1&0\\ 0&0&0&0&1
\\ 1&1&1&1&-1\end {array} \right] 
$$
With the above transformation we can see the new representative is generated by 
$$
 \left[ \begin {array}{cccc|c} -1&0&0&0&0\\-1&0&0&-1
&0\\-1&0&1&0&0\\1&-1&0&0&0
\\ \hline 1&0&0&0&1\end {array} \right] 
\tand
 \left[ \begin {array}{cccc|c} -1&0&0&-1&0\\-1&0&0&0
&0\\-1&1&0&0&0\\1&0&-1&0&0
\\ \hline 1&0&0&0&1\end {array} \right] 
$$
Now by considering $M$ to be the corresponding lattice to 
$$
 \left[ \begin {array}{cccc} -1&0&0&0\\-1&0&0&-1
\\-1&0&1&0\\1&-1&0&0\end {array}
 \right] 
 \tand
 \left[ \begin {array}{cccc} -1&0&0&-1\\-1&0&0&0
\\-1&1&0&0\\1&0&-1&0\end {array}
 \right] 
$$
we can produce
$$
\exactseq{}
.$$
The lattice $M$ corresponds to [4,27,3,1] with subgroups [ 4, 1, 1, 1 ], [ 4, 3, 1, 3 ], [ 4, 27, 1, 1 ], [ 4, 27, 3, 1 ] where all of them are rational. So (5,901,3) is hereditarily rational.

 \section{(5,918,4)}
The group is generated by 
$$
  \left[ \begin {array}{ccccc} 0&-1&0&-1&0\\ \noalign{\medskip}-1&1&0&1
&1\\ -1&0&0&0&0\\ 0&-1&1&0&0
\\ 1&0&0&-1&0\end {array} \right] 
\tand
 \left[ \begin {array}{ccccc} 0&1&0&1&1\\ 0&-1&-1&-1
&-1\\ 0&0&-1&0&-1\\ -1&0&1&1&1
\\ 0&1&0&0&1\end {array} \right] 
$$
The dual group is generated by 
$$
\left[ \begin {array}{ccccc} 0&-1&-1&0&1\\ -1&1&0&-
1&0\\ 0&0&0&1&0\\ -1&1&0&0&-1
\\ 0&1&0&0&0\end {array} \right] 
\tand
 \left[ \begin {array}{ccccc} 0&0&0&-1&0\\ 1&-1&0&0&
1\\ 0&-1&-1&1&0\\ 1&-1&0&1&0
\\ 1&-1&-1&1&1\end {array} \right]
$$
Algorithm (1) produces the change of basis matrix 
$$
  \left[ \begin {array}{ccccc} 0&1&1&-2&1\\ 1&0&0&0&0
\\ 0&1&0&0&0\\ 0&0&1&0&0
\\ 0&0&0&1&0\end {array} \right]
$$
With the above transformation we can see the new representative is generated by 
$$
  \left[ \begin {array}{ccccc} 1&0&1&1&0\\ -1&0&-1&0&0
\\ -2&1&-1&0&0\\ 2&0&1&0&0
\\ -1&0&-1&0&1\end {array} \right] 
\tand
 \left[ \begin {array}{ccccc} -1&-1&-1&-1&0\\ 1&-1&1
&0&0\\ 1&1&2&2&0\\ -1&0&-2&-1&0
\\ 1&0&1&1&1\end {array} \right] 
$$
Now by considering $M$ to be the corresponding lattice to 
$$
\left[ \begin {array}{cccc} 1&0&1&1\\ -1&0&-1&0
\\ -2&1&-1&0\\ 2&0&1&0\end {array}
 \right] 
 \tand
 \left[ \begin {array}{cccc} -1&-1&-1&-1\\ 1&-1&1&0
\\ 1&1&2&2\\ -1&0&-2&-1
\end {array} \right] 
$$
we can produce
$$
\exactseq{}
.$$
The lattice $M$ corresponds to [4,31,1,2] which is a subgroup of $[4,31,7,1]$. In \cite{Nicole1} it is shown that $[4,31,7,1]$ is hereditarily rational and so is (5,918,4).
