\chapter{Lattices of Subgroups in Special Cases}
%%%%%%%Case8
\begin{figure}[H]
\centering
\tikzstyle{every node}=[draw=black,thick,anchor=west]
\tikzstyle{selected}=[fill=gray!30]
\begin{tikzpicture}[%
  grow via three points={one child at (1.5,-1) and
  two children at (1.5,-1) and (1.5,-2)},
  edge from parent path={(\tikzparentnode.south) |- (\tikzchildnode.west)}]
  \node {$G_8$}
    child { node [selected]{$[ 5, 389, 4 ]$}}
    child [missing] {}		
    child { node {$[  5, 607, 4  ]$}
		child{ node[selected]{$[ 5, 98, 28 ]$}}
		child{ node[selected]{$[ 5, 174, 2 ]$}}
		child{ node{$ [ 5, 580, 4 ]$} 
			child{ node[selected]{$[ 5, 19, 14 ]$}}
			child{ node[selected]{$ [ 5, 164, 2 ]$}}		
		} 
		child[missing]{}  
    }
    child[missing]{}
    child[missing]{}
    child [missing] {}
    child [missing] {}	
    child [missing] {}
    child { node {$[ 5, 917, 3 ]$}
		child{ node[selected]{$[ 5, 57, 8 ]$}}
		child{ node[selected]{$ [ 5, 901, 3 ]$}}    
    }
    child[missing]{}
    child[missing]{}
    child { node{$[ 5, 952, 4  ] $} 
		child{ node[selected]{$[ 5, 174, 5 ]$}}
		child{ node{$ [ 5, 580, 4 ]$}
			child{ node[selected]{$[ 5, 19, 14 ]$}}
			child{ node[selected]{$ [ 5, 164, 2 ]$}}			
		}
		child[missing]{}
		child[missing]{}
		child{ node[selected]{$ [ 5, 901, 3 ]$}}    
    }
    child [missing] {}				
    child [missing] {}				
    child [missing] {};
\end{tikzpicture}
\caption{Conjugacy classes of subgroups of $G_8$. Algorithm (1) works for the gray ones.}
\label{G8Lat}
\end{figure}

%%%%%%Case9

\begin{figure}[H]
\centering
\tikzstyle{every node}=[draw=black,thick,anchor=west]
\tikzstyle{selected}=[fill=gray!30]
\begin{tikzpicture}[%
  grow via three points={one child at (1.5,-1) and
  two children at (1.5,-1) and (1.5,-2)},
  edge from parent path={(\tikzparentnode.south) |- (\tikzchildnode.west)}]
  \node {$G_9$}
    child { node [selected]{$[ 5, 389, 4 ]$}}
    child [missing] {}		
    child { node {$[ 5, 607, 9 ]$}
		child{ node[selected]{$[ 5, 98, 28 ]$}}
		child{ node[selected]{$[ 5, 174, 5 ]$}}
		child{ node{$ [ 5, 580, 4 ]$} 
			child{ node[selected]{$[ 5, 19, 14 ]$}}
			child{ node[selected]{$ [ 5, 164, 2 ]$}}		
		} 
		child[missing]{}  
    }
    child[missing]{}
    child[missing]{}
    child [missing] {}
    child [missing] {}	
    child [missing] {}
    child { node {$[ 5, 917, 4 ]$}
		child{ node[selected]{$[ 5, 57, 8 ]$}}
		child{ node[selected]{$ [ 5, 901, 3 ]$}}    
    }
    child[missing]{}
    child[missing]{}
    child { node{$[ 5, 952, 2 ] $} 
		child{ node[selected]{$[ 5, 174, 2 ]$}}
		child{ node{$ [ 5, 580, 4 ]$}
			child{ node[selected]{$[ 5, 19, 14 ]$}}
			child{ node[selected]{$ [ 5, 164, 2 ]$}}			
		}
		child[missing]{}
		child[missing]{}
		child{ node[selected]{$ [ 5, 901, 3 ]$}}    
    }
    child [missing] {}				
    child [missing] {}				
    child [missing] {};
\end{tikzpicture}
\caption{Conjugacy classes of subgroups of $G_9$. Algorithm (1) works for the gray ones.}
\label{G9Lat}
\end{figure}

%%%%%%%Case10

\begin{figure}[H]
\centering
\tikzstyle{every node}=[draw=black,thick,anchor=west]
\tikzstyle{selected}=[fill=gray!30]
\begin{tikzpicture}[%
  grow via three points={one child at (2,-1) and
  two children at (2,-1) and (2,-2)},
  edge from parent path={(\tikzparentnode.south) |- (\tikzchildnode.west)}]
  \node {$G_{10}$}
    child { node {$[ 5, 391, 4 ]$}
		child{ node[selected]{$[ 5, 18, 28 ]$}}
		child{ node{$[ 5, 173, 4 ]$}
			child{ node[selected]{$[ 5, 6, 3 ]$}}
			child{ node[selected]{$[ 5, 164, 2 ]$}}		
		}
		child[missing]{}
	child[missing]{}
		child{ node[selected]{$[ 5, 174, 2 ]$}}
		child{ node{$[ 5, 461, 4 ] $}
			child{ node[selected]{$[ 5, 6, 3 ]$}}
			child{ node[selected]{$[ 5, 164, 2 ]$}}		
		}    
    }
	child[missing]{}
	    child [missing] {}	
    child [missing] {}
    child [missing] {}
    child [missing] {}	
    child [missing] {}	
    child [missing] {}			
    child { node {$[ 5, 608, 4 ]$}
		child{ node[selected]{$[ 5, 99, 57 ]$}}
		child{ node{$[ 5, 173, 4 ]$}
			child{ node[selected]{$[ 5, 6, 3 ]$}}
			child{ node[selected]{$[ 5, 164, 2 ]$}}			
		}
		child[missing]{}
		child[missing]{}
		child{ node{$[ 5, 580, 4 ]$} 
			child{ node[selected]{$[ 5, 19, 14 ]$}}
			child{ node[selected]{$ [ 5, 164, 2 ]$}}		
		} 
		child[missing]{}  
    }
    child[missing]{}
    child[missing]{}
    child [missing] {}
       child [missing] {}
          child [missing] {}
    child { node[selected]{$[ 5, 918, 4 ]$}
    }
    child { node{$[ 5, 952, 2 ] $} 
		child{ node[selected]{$[ 5, 174, 2 ]$}}
		child{ node{$ [ 5, 580, 4 ]$}
			child{ node[selected]{$[ 5, 19, 14 ]$}}
			child{ node[selected]{$ [ 5, 164, 2 ]$}}			
		}
		child[missing]{}
		child[missing]{}
		child{ node[selected]{$ [ 5, 901, 3 ]$}}    
    }
    child [missing] {}				
    child [missing] {}				
    child [missing] {};
\end{tikzpicture}
\caption{Conjugacy classes of subgroups of $G_{10}$. Algorithm (1) works for the gray ones.}
\label{G10Lat}
\end{figure}
\bigskip

%%%%%%%%%Case15

\begin{figure}[H]
\centering
\tikzstyle{every node}=[draw=black,thick,anchor=west]
\tikzstyle{selected}=[fill=gray!30]
\begin{tikzpicture}[%
  grow via three points={one child at (2,-0.8) and
  two children at (2,-0.8) and (2,-1.6)},
  edge from parent path={(\tikzparentnode.south) |- (\tikzchildnode.west)}]
  \node {$G15$}
    child { node [selected]{$[ 5, 81, 54 ]$}  
    }
    child { node {$[ 5, 391, 4 ]$}
		child{ node[selected]{$[ 5, 18, 28 ]$}}
		child{ node{$[ 5, 173, 4 ]$}
			child{ node[selected]{$[ 5, 6, 3 ]$}}
			child{ node[selected]{$[ 5, 164, 2 ]$}}		
		}
		child[missing]{}
	child[missing]{}
		child{ node[selected]{$[ 5, 174, 2 ]$}}
		child{ node{$[ 5, 461, 4 ] $}
			child{ node[selected]{$[ 5, 6, 3 ]$}}
			child{ node[selected]{$[ 5, 164, 2 ]$}}		
		}    
    }
    child[missing]{}
    child[missing]{}
    child [missing] {}
       child [missing] {}
          child [missing] {}
      child [missing] {}
        child [missing] {}
    child { node {$[ 5, 606, 4 ]$}
		child{ node[selected]{$[ 5, 22, 14 ]$}}
		child{ node{$[ 5, 461, 4 ]$}
			child{ node[selected]{$[ 5, 6, 3 ]$}}
			child{ node[selected]{$[ 5, 164, 2 ]$}}			
		}
		child[missing]{}
		child[missing]{}
		child{ node{$[ 5, 580, 4 ]$} 
			child{ node[selected]{$[ 5, 19, 14 ]$}}
			child{ node[selected]{$ [ 5, 164, 2 ]$}}		
		}     
    }
	    child [missing] {}	
    child [missing] {}
    child [missing] {}
    child [missing] {}	
    child [missing] {}	
    child [missing] {}		
     child { node {$[  5, 607, 4  ]$}
		child{ node[selected]{$[ 5, 98, 28 ]$}}
		child{ node[selected]{$[ 5, 174, 2 ]$}}
		child{ node{$ [ 5, 580, 4 ]$} 
			child{ node[selected]{$[ 5, 19, 14 ]$}}
			child{ node[selected]{$ [ 5, 164, 2 ]$}}		
		} 
		child[missing]{}  
    }
     child [missing] {}				
    child [missing] {}				
    child [missing] {}
     child [missing] {}						
     child { node {$[ 5, 608, 4 ]$}
		child{ node[selected]{$[ 5, 99, 57 ]$}}
		child{ node{$[ 5, 173, 4 ]$}
			child{ node[selected]{$[ 5, 6, 3 ]$}}
			child{ node[selected]{$[ 5, 164, 2 ]$}}			
		}
		child[missing]{}
		child[missing]{}
		child{ node{$[ 5, 580, 4 ]$} 
			child{ node[selected]{$[ 5, 19, 14 ]$}}
			child{ node[selected]{$ [ 5, 164, 2 ]$}}		
		} 
		child[missing]{}  
    };
   
\end{tikzpicture}
\caption{Conjugacy classes of subgroups of $G_{15}$. Algorithm (1) works for the gray ones.}
\end{figure}